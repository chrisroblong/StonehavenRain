As a part of a complex system,
    any weather event is necessarily multifactorial,
    with no single cause.
It is therefore necessary to use a probabilistic lens to perform a scientific analysis.

In this case, the weather at a given place and time of year is expressed as a statistical distribution of events.
The climate may then be considered either the distribution itself
    or the parameters informing the distribution.
In other words, ``Climate is what we expect, weather is what we get.''~\cite{Herbertson_1935}.

The scientific method may then be applied,
    using the climate distribution as a dependent variable to the factor being considered.
Issues of both a technical and a theoretical nature can arise with this approach
    and will be discussed where they may arise in the process.

In all the examples given in this section,
    the independent variable is anthropogenic climate change.
This is often quantified by using a temperature metric,
    although additional factors may be considered.

It may be noted that
    climate event attribution techniques may be of use outside risk analysis from modern changes in the climate
    and generalised to cover large-scale events due to other factors changing in the climate.
For example,
    Ruddiman~\cite{Ruddiman_2010} has attempted to attribute the lack of a recent glaciation to the beginning of agriculture.

An event attribution study aims to provide two quantities.
The first is the change in the probability of the event.
The second is the change in the intensity of the event.

\subsubsection{Development of Event Attribution}

In 2003, whilst experiencing flooding as a result of extreme weather,
    Allen~\cite{Allen_2003} posited the probabilistic interpretation of climate change impacts,
    proposing a potential to find greenhouse gas emitters liable for extreme weather events.
A notable early use case of this probabilistic approach was by Stott et al.~\cite{Stott_2004},
    finding that 75\% of the risk of the European heatwave is attributable to anthropogenic climate change.

World Weather Attribution produces rapid Event Attribution studies,
    some of which are within a month of the event occurring~\cite{van_Oldenborgh_et_al_2018},
    using the probabilistic approach.
Alternative approaches also exist.
One of these is the `Boulder' methodology,
    described in section 2.2 of~\cite{Otto_2017},
    which seeks to break down the proportions to which the event was caused by natural variability or Anthropogenic Climate Change.
Another is the `storyline' methodology,
    outlined by Shepherd et al.\ in~\cite{Shepherd_et_al_2018},
    although this has been considered contentious~\cite{García-Portela_Maraun_2023}.

\section{Steps to Event Attribution}\label{sec:attrsteps}

The precise number of steps for successful event attribution varies by the situation.
Van Oldenborgh et al.~\cite{van_Oldenborgh_et_al_2021} give 8 separate steps,
    while Otto~\cite{Otto_2017} gives 6 steps.
Both of these methodologies are functionally identical;
    I shall base my description upon four of the steps described by Otto.

Only four steps are given in full below.
This is due to the removal of an analysis trigger step at the start of the process
    and a communication step at the end of the process,
    both of which I will cover here briefly.

An analysis trigger is a set of criteria used to determine whether an event should be investigated.
These serve a utilitarian purpose,
    allowing the events with the greatest impact to be investigated first.
These may include the number of deaths or number impacted,
    as in chapter 2 of~\cite{van_Oldenborgh_et_al_2021},
    or economic and cultural factors, as in~\cite{Tett_Soon}.
The former criteria may skew analysis towards poorer nations~\cite{Kahn_2005},
    while the latter may skew analysis towards richer nations.

It is of note that the potential liability of anthropogenic climate change may be a triggering factor,
    so the analysis begins due to a hypothesis that the event was caused by anthropogenic climate change.
I believe that this may bias the results of a meta-analysis of event attribution studies,
    and so increase the challenge in determining anthropogenic climate change as a cause of more extreme events in general.
I omit to give any further detail on the analysis trigger as it is beyond the scope of mathematical or physical interest.

The communication of the outcome of the analysis may also be considered in the study.
This is especially true outside the scientific community
    and considers the risk of the audience drawing incorrect conclusions,
    as well as the risk of policymakers acting on incorrect information.

\subsection{Event Definition}\label{subsec:backeventdef}

To perform an attribution analysis,
    the weather event must first be defined in clear and quantitative terms.
This requires specifying a variable with a threshold that has been crossed,
    as well as the time and region at which an event may be considered.

This is not necessarily straightforward.
An event attribution is normally triggered by the event's social or environmental impact,
     not just due to the meteorological significance.
Therefore, care must be taken to ensure that the definition of the event is sufficiently similar to the conditions that would cause a similar impact.

Many of the challenges in Event Attribution arise directly from a poor Event Definition.
Events that are too narrowly defined not be reproducible by climate models,
    while events that are too wide will give results for more minor events that would not have had the same impacts as the actual event.

\subsubsection{Variable and Threshold}

The climate variable used to define the event requires travelling up the causal chain that leads to it.
As a set of examples,
    floods/droughts are caused by extreme highs/lows in rainfall,
    while heat waves and cold spells are caused by extremely high or low temperatures.

However,
    all of these events require extreme rainfall/temperature to be sustained for the event to be hazardous.
Simple metrics may be applicable here,
    as the definition can then use the rainfall/temperature averaged over a given time period.
The selection of the time period itself also requires a consideration of the event.
A damaging heavy storm requires extreme winds over a matter of minutes,
    while it may take days for the land to dry out sufficiently for a drought to occur.

More advanced metrics may be used.
For example, an Expert Team on Climate Change Detection Indices~\cite{Zhang_2011}
    describes a `Growing Season Length',
    allowing climate-caused famine to be quantified using only daily temperature data.
A `Wet-Bulb Temperature' has been used to combine humidity and temperature data to create an index that better measures the effect of a heatwave on human health~\cite{Li_2020}.

\subsubsection{Spatial and Temporal Region}

In addition to defining a threshold variable,
    it is necessary to define the regions and times at which an event can be considered sufficiently similar to that being analysed.
This is due to the fact that both historical trend analyses are impossible on a single extreme event at a single place on a single day of the year
    and that climate models will rarely reproduce events in such a narrow definition.

For the spatial region,
    a region with a known climate similar to that in which the event occurred is chosen,
    typically directly around the location of the event.
This is done so that the frequency and types of events will match the location of the original event,
    as well as to ensure that the factors that change the climate
    are similar to those changing the climate at the point of the event.

In addition to the aforementioned similarity in location,
    a historical trend analysis can aid in defining a region,
    along with tools including topography and climate classifications.
The region is chosen based on similar weather,
    not a similarity in hazard risk,
    as the goal is to perform an attribution for the event,
    not to advise on potential future risk.
For example,
    a region defined for a flood event over a flood-prone area would only consider areas with similar rainfall,
    not those similarly flood-prone.

For the temporal region,
    or time period for the event to be considered,
    an analogous process must be used.
It is often appropriate to use a seasonal time period,
    as this can also limit the event to those with similar meteorology,
    such as by only considering monsoon rainfall~\cite{Otto_et_al_2023}.
This can also prove to be a practical step,
    reducing the amount of data to be used later in the attribution process.

Additional `Boundary Conditions'~\cite{van_Oldenborgh_et_al_2021} may then also be imposed.
Of note is the importance of  El Niño,
    especially for drought events in the southern hemisphere~\cite{Lyon_2004},
    so this may also be accounted for when defining these events.

Once all of these definitions have been made,
    a `class' of events has been described.
This class will then be used to find similar events in both the historical and model data.

\subsection{Model Evaluation}\label{subsec:backmodeleval}

For the use of the model to be justified,
    it must accurately represent the phenomena being studied.
This can involve three steps~\cite{van_Oldenborgh_et_al_2021}.

First, the model must be capable of representing the event.
If the area or time period of the event,
    or the dynamics that drive it, are too small to be represented within the resolution of the model,
    the model cannot be used for an event attribution analysis.

Second is that the statistics of the model match with the statistics of the empirical (observed) extreme event.
This can be done by fitting relevant distributions to both the empirical and model data sets and ensuring that the
    parameters governing the behaviour of the tail of the distributions (the extremes) are comparable.
A visual method is also possible, comparing the model data points to the distribution fit to the empirical data.
This can be done with a return plot (\cite{van_Oldenborgh_et_al_2021} fig. 4)
    or with a quantile-quantile plot (as seen in~ Coles' textbook\cite{Coles_2001} fig 3.6. Note that the `empirical and model labels
    would be reversed as it is the data points of the model that are being compared to a fit to the empirical data points.)

Finally,
    the physical processes underlying the extremes in the model must match with observations.
An example of this is that the frequency of a southern hemisphere drought event maintains its correlation with the El Niño phase,
    as described in subsection~\ref{subsec:backeventdef}.

\subsection{Likelihood Estimation and Interpretation}\label{subsec:backlikeest}

To apply Allen's~\cite{Allen_2003} probabilistic interpretation,
    the model can be used to generate datasets for the world with and without warming.
With a sufficient number of models run,
    a distribution can be fit to each of these scenarios.

The probabilities of the event threshold being crossed are then found,
    with the probability in the world without warming $p_0$ and the probability in the distribution with warming $p_1$.
From these, a probability ratio  $p_1 / p_0$can be computed,
    showing the growth in the likelihood of the extreme event due to climate change.

Alternatively,
    if the return period of the extreme event is known,
    the intensity of events corresponding to that return period in the distributions with and without warming can be found,
    with the difference between these being $\delta I$.
For rainfall events, it is possible to calculate an intensity ratio $I_1 / I_0$, where $I_1$ is the intensity of the event
    with warming and $I_0$ is the intensity of the event without warming.
This shows how the intensity of the rainfall event scales with climate change.

The analysis in this report creates distributions with different levels of warming from a single model,
    which contains different levels of warming
    and computing the effect that the warming within the model has on the distribution of extrema.

\section{Extreme Value Statistics}\label{sec:exstats}

This subsection will describe the statistical foundations of extreme event modelling,
    paraphrasing Coles' book~\cite{Coles_2001} unless where otherwise stated.
Climate events that pose hazards are necessarily extreme.
Therefore, to model these events,
    extreme value distributions are necessary.

Throughout the analysis,
    we assume that the variables are independent and identically distributed ($i.i.d.$).
For annual extrema,
    this is almost always true,
    as the events are typically a year apart and so are not influencing one another.

For event definitions that average a variable over a season,
    as opposed to focusing on the extreme
    a normal distribution is instead appropriate.
This is because the Central Limit Theorem states that the limit of the mean of $n$ $i.i.d.$ random variables tends to a normal distribution.
The equivalents for extreme events, the GEV distribution and Extremal Types Theorem
    will be explained in due course.

\subsection{GEV Distribution}\label{subsec:gev}

\subsubsection{Extremal Types Theorem}

This explanation is given in section 3.1 of~\cite{Coles_2001}.

Let $M_n$ be the maximum of $n$ $i.i.d.$ random variables.
If there exist sequences of constants $a_n > 0$ and $b_n$, and non-degenerate $G$ such that:

\[ P\left( \frac{M_n - b_n}{a_n}  \leq z \right) \rightarrow G(z) \text{ as } n \rightarrow \infty \]

Then $G$ is a member of the GEV (Generalised Extreme Value) family.

Section 3.1.4 of~\cite{Coles_2001} outlines a proof of the Extremal Types Theorem.
This involves showing that being a GEV distribution is equivalent to being `Max-Stable',
    then that the maxima of GEV distributions are Max-Stable,
    giving the GEV as the distribution of the original maximum in the infinite limit.
It is of note that this is similar to one approach to proving the Central Limit Theorem,
    which considers the normal distribution as a unique fixed point under convolutions,
    resulting in normal distributions being an infinite limit of the sum or mean~\cite{Hamedani_Walter_1984}.

Section 3.5 of~\cite{Coles_2001} shows, \textit{mutatis mutandis},
    that the extremal types theorem applies identically to minima by reversing the sign of the data.

\subsubsection{Parameters and Types of GEV distributions}

The GEV distribution has a CDF given by:
\begin{equation}\label{eq:gevcdf}
    G(z) = \exp \left( - \left( 1 + \xi \left( \frac{z-\mu}{\sigma} \right)  \right)^{-\frac{1}{\xi}} \right)
\end{equation}

This has three parameters, the location $\mu$, the scale $\sigma$ and the shape $\xi$.
It is clear from this CDF that the change of location and scale parameters represent moving and rescaling the distribution,
    much as the mean and standard deviation parameters do for a normal distribution.

$1-G(z)$ Is the survival function of the distribution.
This function gives the probability of an interval containing an event more extreme (of larger intensity) than $z$.

The shape parameter determines the thickness of the tail of the distribution.
The larger the shape parameter is, the thicker the tail of the distribution.
This is visible in Figure~\ref{fig:gevshape}.

Where $\xi > 0$,
    the distribution is heavy-tailed.
This type of GEV distribution is also known as a Fréchet distribution.
Special cases of this distribution have unique properties.
Where $\xi \geq \frac{1}{2}$ the distribution has infinite variance and
    where $\xi \geq 1$ the distribution has infinite mean.

Where $\xi < 0$, the distribution is bounded,
    also known as a reversed Weibull distribution.
These distributions are limited to a maximum `most extreme' value of $\mu - \frac{\sigma}{\xi}$.
It can therefore be expected that distributions modelling extrema with physical constraints would have a negative shape parameter.
For example, Table 3 of Chikobvu and Chifurira's paper~\cite{Chikobvu_2015} modelling rainfall minima finds a negative shape parameter,
    which would be expected as rainfall cannot be negative.

Where $\xi = 0$,
    the distribution is a Gumbel distribution.
The CDF of this distribution requires taking the limit as $\xi \rightarrow 0$,
    giving a double exponential CDF\@.
The extrema of normal distributions converge to a Gumbel distribution,
    shown as a step in the proof of Proposition 1 of~\cite{Bailey_2014}.
This means that, with the Central Limit Theorem,
    the maxima of a sample of means of any distribution approach a Gumbel distribution as both samples grow larger,
    provided the samples are independent.

The above paragraph suggests that a Gumbel distribution is appropriate for modelling rainfall maxima,
    as the rainfall is averaged over a time period and then a maximum is taken over those time periods for a given year or season.
However, this has not been found to be true.
Koutsoyiannis~\cite{Koutsoyiannis_2003} finds both an ``extremely slow'' theoretical convergence to a Gumbel distribution, and
    that the Gumbel distribution does not fit empirical rainfall maxima.
Therefore, the three-parameter GEV distribution is used to model rainfall maxima in event attribution.

\subsubsection{Return levels of GEV distributions}

Equation~\ref{eq:gevcdf} can be inverted to give a return level $z_p$ for a probability $p$, given in~\cite{Coles_2001}:
\begin{equation}\label{eq:gevreturn}
    z_p = \mu - \frac{\sigma}{\xi}\left( 1-\left( -\log\left( 1-p \right) \right)^{-\xi} \right)
\end{equation}
This is otherwise known as the inverse survival function,
    where $z_p$ is the value for which the probability of finding a value greater than $z_p$ is $p$.
The critical value for a given probability $p$ breaks down into the sum of two components,
    the location and a quantity that is proportionate to the scale.

The rarity of events for a return period $P$ can be computed,
    as the probability $p$ is equal to $1/P$.
This allows the GEV distribution to use the language common for extreme weather events.
With knowledge of the underlying distribution,
    it is possible to compute the intensity of an event for any return period.

\subsection{GEV Parameter Estimation}\label{subsec:parameterest}

In sections~\ref{subsec:radardatafit} and~\ref{subsec:radardatafit},
    parameters are fitted to data using the extRemes R library~\cite{extremes_R}.
This software uses MLE (Maximum Likelihood Estimation) to get the best estimates of the parameters of a dataset,
    assuming the data obeys a GEV distribution.
This technique will be covered briefly here.

The extRemes package is also used to fit linear changes in a parameter with respect to a covariate.
Although this uses the same underlying mathematics as this section,
    relying on MLE,
    the exact algorithm used is beyond the scope of this report.

\subsubsection{Maximum Likelihood Estimation}

For a set of data points to which a GEV distribution is to be fitted,
    there is a likelihood function $L(\mu, \sigma, \xi)$,
    defined as the product of the PDF of the distribution at each data point.
Related is the log-likelihood,
    which may be easier to work with as it provides a sum of the log-likelihood of each data point.

The Maximum Likelihood Estimators are the values of the parameters that maximise the likelihood function.
These values are then taken as the parameters of the distributions.

Multivariable calculus is used to find the Maximum Likelihood Estimators.
In some cases, this can be done analytically,
    although in others numerical methods must be used.

Section 3.3 of Cole's textbook~\cite{Coles_2001} covers the properties of Maximum Likelihood Estimators for the GEV distribution,
    as well as inference for return levels.

\subsubsection{Akaike Information Criterion}

To assess the goodness of fit,
    the Akaike Information Criterion (AIC)~\cite{AIC_1974} is used.
This is given in the following equation:
\begin{equation}\label{eq:AIC}
    AIC = 2k - 2\ln \left( L \right)
\end{equation}
Where $k$ is the number of parameters and $L$ is the value of the likelihood function with the data.

A better fit has a lower AIC.
Equation~\ref{eq:AIC} imposes a cost of 2 per parameter,
    preventing overfit models with many parameters being considered a good fit for the data.

For Models 1 and 2, the probability that Model 2 loses less information than Model 1,
    i.e.\ that Model 2 is a better fit than Model 1,
    is given in~\cite{AIC_Info}:
\begin{equation}\label{eq:AIC_Info}
    \exp \left( \frac{1}{2} \left( AIC_1 - AIC_2 \right) \right)
\end{equation}
