\section{Results}\label{sec:results}

\begin{comment}
This section should detail the obtained results in a clear,
easy-to-follow manner. It is important to make clear what are original
results and what are repeats of previous calculations or computations.
Remember that long tables of numbers are just as boring to read as
they are to type-in!

Use graphs to present your results wherever practicable.

Results or computations should be presented with uncertainties
(errors), both statistical and systematic where applicable.

Be selective in what you include: half a dozen \emph{e.g.}~tables that
contain wrong data you collected while you forgot to switch on the
computer are not relevant and may mask the correct results.
\end{comment}

\subsection{Radar Data Fit}\label{subsec:radardatafit}

\begin{figure}[H]
    \centering
    \includegraphics[width=150mm]{radarreturnplot}
    \caption{A line graph plotting the return period in years against the intensity of the event,
        taken from fitting a GEV distribution to the event distribution.
    The dashed purple vertical line is the intensity of the Stonehaven event.
    The solid red vertical line is the actual one-hour rainfall maximum at the Stonehaven crash location.
    The grey area gives the uncertainties from a Monte Carlo bootstrap of approximately 1000 samples.}
    \label{fig:radarreturnplot}
\end{figure}

It was found that the maximum hourly rainfall at the crash location was 18.2mm/hr and
    that the empirical return period for the Stonehaven event was 17 years,
    as can be seen in figure~\ref{fig:radarreturnplot}.
The intensity of the Stonehaven event,
    the 0.95 quantile of the radar grid cells having their 2020 seasonal maxima in the AM of the 12th of August,
    was found to be 25.2mm/hr.

Figure~\ref{fig:radarreturnplot} was created using the survival function of the GEV distribution fitted to the event data,
    getting the return period as the reciprocal of the probability.
This had parameters location $\mu = 10.7$, scale $\sigma = 4.26$ and shape $\xi = -0.173$.

\subsection{Model Data}\label{subsec:modelcorr}

\subsubsection{Model Correlations}

\begin{figure}[H]
    \centering
    \includegraphics[width=150mm]{2cet_scatter}
    \caption{Scatter diagrams, from the Convection Permitting Model,
        for summer CET vs Stonehaven region saturated humidity (g/Kg) (a),
        Regional Temperature (b),
        Regional  Precipitation (mm/day) (c) and
        spatial median Rx1hr (mm/hr) for the region.
    The region is all points within a square of about 100x100km centered on the Stonehaven crash location.
    Colors indicate the ensemble number, black line is the best-fit regression slope.
    Text box shows best estimate and standard error in the estimated regression.
    Dotted black vertical lines show (left to right) CET summer mean for 1850--1899, 2021 and estimated CET at +2K warming.
    Title shows $R^2$ correlation coefficients for fit.
    Dotted vertical red lines show minimum and maximum CET values from 1850--2020 observations.}
    \label{fig:2cet_scatter}
\end{figure}

It is unsurprising that the correlations found in (a) and (b) are very similar,
    with both having $R^2$ values that round to the same number to the nearest percentage.
Some data points appear in almost identical points in both scatter diagrams.
This is evident from equation~\ref{eq:qsat}.
By differentiating at $T=18$,
    equation~\ref{eq:qsat} has Taylor series to first order of $20.5627+1.2863(T-18)$.
By substituting $T=14$ and $T=22$ into this linear approximation,
    only small errors of $0.532$ and $0.603$ are found.
This suggests that in the CET range, equation~\ref{eq:qsat} is effectively linear and so the fit onto the Stonehaven Region saturated humidity
    is a linear scaling of the fit onto the Stonehaven Region Temperature.

(b) finds that the temperature in the Stonehaven Region is expected to increase by 0.79 degrees for every increase
    in CET.
This is of interest, as CET itself is expected to increase by 0.96 per degree of climate warming~\cite{Tett_Soon},
    so the Stonehaven Region temperature is expected to increase far slower than global temperatures.

In (c),
    a negative correlation is found between the mean rain in the Stonehaven region and CET,
    suggesting that total rainfall decreases with an increase in temperature.
This contrasts with the positive correlation between the mean monthly maximum rainfall found in (d),
    which suggests that the scaling of the extreme distribution will be positive.

\subsubsection{Model GEV fit}

\begin{figure}[H]
    \centering
    \includegraphics[width=100mm]{2cpm_gev_fit}
    \caption{a) Location parameter $\mu = \mu_0 + \overline{T_{2012-2021}}\mu_1$  for 2012--2021 CET and Rx1hr.
    b) Scale parameter $\sigma = \sigma_0 + \overline{T_{2012-2021}}\sigma_1$ for 2012--2021 CET and Rx1hr.
    c) Scatter plot of d(location)/d(CET) vs d(scale)/d(CET) as of 2012--2021 parameters.
    Dashed lines show expected changes if changes predicted by Clausius-Clapeyron relationship.
    Dots are colored by topography.
    Large red dot shows mean over all points where height is more than 0m and less than 400m.
    Red ellipse shows 95\% uncertainty for this mean, see subsection~\ref{subsec:riskratios}.
    Other colored large dots and ellipses show similar means and uncertainty ellipses for Rx2hr (brown), Rx4hr (purple) and Rx8hr (blue).
    Black ellipse shows 95\% uncertainty for all Rx1hr points where height is less than 0m and more than 200m.
    d) as c) but for case with shape parameter varying with CET
    e) QQ-plot for GEV fit to CPM data using CET as a covariate at the Stonehaven Crash location.
    f) as e) but for point 25km west of the Stonehaven Crash Location.}
    \label{fig:2cpm_gev_fit}
\end{figure}

The thin, thick and dashed lines on (a) and (b) represent 0m, 200m and 400m contours respectively.
These charts show that the rainfall is expected to be greater over higher areas,
    with the extreme rainfall distribution having both greater location and scale parameters,
    which are a factor of approximately 1.5x greater over 400m than they are between 0m and 200m.
The rainfall over the sea is extremely low.
This should be disregarded for reasons given in subsection~\ref{subsec:preprocess}

The colours of the dots in (b) show that the scaling of both the shape and scale parameters with CET
    increases with height.
The confidence ellipses show that there is far greater uncertainty about the increase in the location parameter
    than the scale parameter.
The similarity of (b) and (c) suggest that fitting the shape parameter as to the covariate does not have
    a significant effect on the fit of the other two parameters to the covariate.
Table~\ref{tab:CCtable} gives the location of the red, brown, purple and blue dots relative to the
    dashed lines representing Clausius-Clapeyron scaling.

(e) shows that the quantiles for the GEV fit are similar at the Stonehaven Crash location,
    while (f) is taken from a point near the 400m contour.
At this point, the GEV fit breaks down at the top end,
    with the GEV fit expecting values far greater than those provided by the CPM model it is fit to.

\begin{table}[H]
    \centering
    \begin{tabular}{c c c c c}
        Covariate\textbackslash Data & Rx1h & Rx2h & Rx4h & Rx8h \\
        None &7498&6533&5341&4130 \\
        CET &7433&6482&5305&4109 \\
        CET w/ Shape &7434&6483&5305&4109 \\
        CPM Region &7402&6457&5287&4097 \\
        CPM Region w/ Shape &7403&6458&5287&4097 \\
        Stonehaven Region &7402&6457&5287&4098 \\
        Stonehaven Region w/ Shape &7403&6458&5288&4098 \\
        Stonehaven Humidity &7405&6459&5289&4099 \\
        Stonehaven Humidity w/ Shape &7406&6460&5290&4100 \\
    \end{tabular}
    \caption{AIC for maximum summer 1, 2, 4 and 8 hourly summer maximum rainfall
        above 0m and below 400m in the Stonehaven Region (+/- 0.5 degrees of the Stonehaven Crash location)
        GEV fits with different covariates.
    The covariates are CET, the average temperature in the CPM region (UK),
    the average temperature in the Stonehaven Region and
    the Saturation Humidity in the Stonehaven Region}
    \label{tab:AICtable}
\end{table}

The values of AIC for different rain intervals are not directly comparable,
    as the fit is to different datasets.
For all covariates,
    it is found that fitting the shape to the covariate had no significant effect on the goodness of the GEV fit.
For the 1 hour maxima,
    fitting the GEV with the CPM Region temperature, Stonehaven Region Temperature and Stonehaven Region Humidity gave almost equally good fits.
It was expected that the fit to the Stonehaven Region Temperature and Humidity would be similar for
    reasons given in the discussion of~\ref{fig:2cet_scatter},
    although Humidity does perform slightly worse as a covariate.
Having no covariate provides a worse fit for the data.

The probability of information loss formula, equation~\ref{eq:AIC_Info},
    can be applied to the AIC values in table~\ref{tab:AICtable} for the one hour maximum model rainfall.
This finds that a fit to CET is of order $10^{-7}$ times as probable to be a better fit than to another covariate,
    and that a fit to no covariate is of order $10^{-21}$ times as probable to be a better fit than to a covariate other than CET.

\begin{table}[H]
    \centering
    \begin{tabular}{c c c}
        Dataset\textbackslash Parameter  & Location $\mu$ & Scale $\sigma$ \\
        Rx1hr &0.98&1.92 \\
        Rx2hr &0.80&1.67 \\
        Rx4hr &0.59&1.33 \\
        Rx8hr &0.38&1.01 \\
    \end{tabular}
    \caption{Ratio of mean parameter scalings to Clausius-Clapeyron scaling, $\left( \frac{\alpha_1}{\alpha_0} \right) / \left( \frac{H_1}{H_0} \right)$,
        for parameter $\alpha$ in the GEV fit to CET and saturated humidity $H$ in a linear fit to CET,
        where the shape parameter is fixed for the fit.
        Subscript $0$ is the value in the average summer CET of 2012--2021,
            subscript $1$ is the increase for every one degree increase of CET.
    Clausius-Clapeyron scaling is 5.73\%.}
    \label{tab:CCtable}
\end{table}

The scaling of all the parameters in table~\ref{tab:CCtable} are positive,
    as would be expected from figure~\ref{fig:2cet_scatter} (d),
    which shows that the maximum rainfall extrema increase with CET .
For the one-hour rainfall maxima,
    the location is scaled almost exactly as would be expected with Clausius-Clapeyron,
    shown by the red dot in figure~\ref{fig:2cpm_gev_fit} (c) being almost exactly on the vertical dashed line,
    while the scale parameter is scaled almost double that expected by Clausius-Clapeyron.
Figure~\ref{fig:2cet_scatter} (a) and (d) also suggest this pattern,
    as the both the saturated humidity and 1 hour maximum rainfall increase by around 6 (ignoring units)
    per degree increase in CET, while saturated humidity takes a value at around 15 for a typical CET,
    with Rx1hr taking a value of 10 and so having a larger proportional increase per degree of warming.

As the time period of the extrema increases,
    the scaling of the parameters decrease.
Figure~\ref{fig:2cet_scatter} (c) suggests that this is to be expected,
    as non-extreme rainfall in the Stonehaven decreases with CET and so by averaging over a longer time period,
    the maximum rainfall exhibits behaviour more similar to average rainfall.

The scale parameter $\sigma$ scales by more with CET than the location parameter $\mu$.
The effect of this is that the modelled maximum rainfall in a typical year does not increase by a large amount,
    but for more extreme years, those with higher return periods,
    the maximum rainfall increases by a larger amount with CET than would be expected with Clausius-Clapeyron.

\subsection{Risk Ratios}\label{subsec:riskratio}

\begin{figure}[H]
    \centering
    \includegraphics[width=\linewidth]{2probradarcpm}
    \caption{Intensity ratio increase (\%) as a function of the return period (a) and
    probability ratio as a function of regional Rx1hr (b).
    The best estimate (line) and 5--95\% uncertainty range (shading) are shown.
    Horizontal dotted lines in a) show expected intensity change if extremes scale with Clausius-Clapeyron.
    The top axis in a and b shows the equivalent rainfall and return period estimated from the radar rainfall data while
    vertical purple line shows the regional rainfall maximum for 2020.
    Dotted, dashed and dot-dashed lines in a and b show results when scaling estimated from simulated 2, 4 and 8 hourly summer maxima (Rx2hr, Rx4hr and Rx8hr).
    c) Intensity ratio increase (\%) as a function of return period for best estimates using different quantiles (labels on PI+2K line) to define regional extreme.
    Hexagon markers show 95\% quantile used in a and b.
    d) As c) but for probability ratio as a function of Rx1hr.
    Vertical purple line shows Stonehaven Crash Rx1hr for 2020-08-12.
    Note these are different from the regional 95\% quantiles shown in a and b.}
    \label{fig:2probradarcpm}
\end{figure}

Figure~\ref{fig:2probradarcpm} (a) shows that the intensity ratio decreases when fitted to model extrema over longer intervals,
    only being sub-Clausius-Clapeyron when the event distribution is scaled like the average over 8 hour model maxima;
(b) shows the same decreasing in risk ratios for scaling like longer-interval model maxima.
This is to be expected from the data in table~\ref{tab:CCtable},
    as it is known that maxima over longer periods give a lower scaling.

(c) Shows that there is little difference in the intensity ratio by defining the Stonehaven Event by a different
    quantile, with all the intensity ratios super-Clausius-Clapeyron and within the confidence interval
    for the intensity ratio of the 0.95 quantile visible in (a).
(c) also shows that the most unusual extrema of the Stonehaven Event were at the 0.5 and 0.8 quantiles,
    both being rarer than 1-in-20 year events, as opposed to the 0.95 quantile definition used in the
    rest of the analysis, which was a 1-in-17 year event.

(d) shows that the probability ratio decreases with the quantile chosen for event definition from the 0.5 quantile onwards,
    going from the 0.5 quantile having a probability ratio of almost 3,
    while the 0.95 quantile has a probability ratio of around 1.8, visible in from the solid red line (b).
Additionally, (d) shows that the Stonehaven Crash location had maximum rainfall between the 0.5 and 0.8 quantiles of the Stonehaven Event,
    not at the 0.95 quantile.

\begin{figure}[H]
    \centering
    \includegraphics[width=\linewidth]{2probradarcpmshape}
    \caption{As for figure~\ref{fig:2probradarcpm},
    with the shape parameter also allowed to vary with CET.}
    \label{fig:2probradarcpmshape}
\end{figure}

Figure~\ref{fig:2probradarcpmshape} (c) and (d) show no difference from (c) and (d) of~\ref{fig:2probradarcpm}.
This allows it to be concluded that fitting the shape parameter of the Rx1hr model data as to a CET covariate has
    no impact on the intensity or probability ratio,
    independent of the choice of quantile used to define the intensity of an Event.

(a) and (b) show the same effects as (a) and (b) of~\ref{fig:2probradarcpm} at the return period and intensity
    of the Stonehaven Event,
    with the ratios decreasing with the time interval of model maxima fit to,
    except with broader (weaker) confidence intervals.
However,
    for events with a return period of more than 25 years or an intensity of more than 30mm/hr,
    the effect starts to reverse,
    with the fit to the 8-hour model maxima giving the largest ratios.
The pattern in (a) and (b) of figure~\ref{fig:2probradarcpm} is completely reversed by return periods of
    more than 45 years or intensities of more than 35mm/hr,
    with both the intensity and probability ratios increasing with the time interval of the maxima.

\begin{table}[H]
   \centering
    \begin{tabular}{c c c}
        Time Period & Intensity Ratio (\%) & 5--95\% uncertainty (\%) \\
        1980s & 2 & 2--3 \\
        2012--2021 & 9 & 11--7 \\
        PI+2K & 18 & 14--22 \\
        With varying shape: && \\
        1980s & 2 & 2--3 \\
        2012--2021 & 9 & 11--7 \\
        PI+2K & 18 & 13--22 \\
    \end{tabular}
    \caption{A table with the intensity ratios $I_1/I_0$ for three different time periods,
        both with and without a covariate shape parameter.
    $I_1$ is the intensity of the event in the Time Period.
    $I_0$ is the intensity of the event Pre-Industrial (1850-1899).}
    \label{tab:irtable}
\end{table}

\begin{table}[H]
   \centering
    \begin{tabular}{c c c}
        Time Period & Probability Ratio & 5--95\% confidence interval (\%) \\
        1980s & 1.10 & 1.06--1.15 \\
        2012--2021 & 1.37 & 1.22--1.61 \\
        PI+2K & 1.78 & 1.46--2.34 \\
        With varying shape: && \\
        1980s & 1.10 & 1.06--1.15 \\
        2012--2021 & 1.37 & 1.23--1.62 \\
        PI+2K & 1.79 & 1.47--2.38 \\
    \end{tabular}
    \caption{A table with the probability ratios $p_1/p_0$ with confidence intervals for three different time periods,
        both with and without a covariate shape parameter.
    $p_1$ is the probability of the event in the Time Period.
    $p_0$ is the probability of the event Pre-Industrial (1850-1899).}
    \label{tab:prtable}
\end{table}

Tables~\ref{tab:irtable} and~\ref{tab:prtable} show that,
    as expected from the parameters scaling positively with CET,
    the both the intensity and probability of the event increase for later time periods,
    when CET is greater.
The ratios represent the intersections of the solid lines with the purple vertical lines in (a) and (b)
    of figures~\ref{fig:2probradarcpm} and~\ref{fig:2probradarcpmshape},
    with the confidence intervals representing the intersection of the shaded areas with the purple vertical line.

Both the ratios and confidence intervals show either no or very little difference that

\begin{table}[H]
    \centering
    \begin{tabular}{c c c}
        Time Period\textbackslash Parameter & Location $\mu$ & Scale $\sigma$ \\
        PI & 10.2 & 3.90 \\
        1980s & 10.4 & 4.02 \\
        2012--2021 & 10.8 & 4.34 \\
        PI+2K & 11.3 & 4.79
    \end{tabular}
    \caption{A table showing the parameters of the GEV distribution for the intensity an event defined in subsection~\ref{subsec:radarprocess} after
        applying the scaling of the location $\mu$ and scale $\sigma$ parameters to the temperatures of different time periods.}
    \label{tab:0.95params}
\end{table}

The data in table~\ref{tab:0.95params} does not have a covariate shape,
    using the fixed shape parameter $\xi = -0.173$ from the empirical fit.
The shape parameter is in the $-0.174$--$-0.172$ range for all three fits with covariate shape,
    so it effectively duplicates the parameters here.

As the shape parameter is negative,
    it is possible to compute hypothetical the maximum,
    infinite return period intensity of an event using the formula $\mu - \frac{\sigma}{\xi}$.
This gives 32.7, 33.6, 35.9 and 39.0mm/hr as hypothetical maxima in
    PI, the 1980s, 2012--2021 and PI+2K Time Periods respectively.

\begin{figure}[H]
    \centering
    \includegraphics[width=150mm]{scaledgevdists}
    \caption{A line chart of the Probability Density Functions of the GEV distributions with parameters given in table~\ref{tab:0.95params},
        with a shape parameter $\xi = -0.172$.
    The Time periods are Pre-Industrial (PI), the 1980s, 2012-2021
    X-axis gives the 0.95 quantile of the maximum summer rainfall by grid cell in mm/hr,
        Y-axis gives the probability density in probability/(mm/hr).
    Vertical purple line is the intensity of the Stonehaven Event.}
    \label{fig:scaledgevdists}
\end{figure}

Figure~\ref{fig:scaledgevdists} illustrates the probability ratios in the top half of table~\ref{tab:prtable}.
$p_0$ is the area to the right of the purple line between 0 and the blue curve,
    while $p_1$ is the area to the right of the purple line between 0 and the yellow, green and red curves for
    the 1980s, 2012--2021 and a world 2K warmer than pre-industrial respectively.


\section{Discussion}\label{sec:discussion}

\begin{comment}
This section should give a picture of what you have taken out of your
project and how you can put it into context.

This section should summarise the results obtained, detail conclusions
reached, suggest future work, and changes that you would make if you
repeated the project.
\end{comment}

\subsection{Model Resolution}\label{subsec:dismodeldef}

Mention~\cite{Kendon_Fischer_Short_2023}.

\subsubsection{Model Fits}

\subsubsection{Risk Ratios}

\subsection{Choices in analysis}\label{subsec:diseventdef}

%  TODO mention lack of trend analysis due to lack of radar data

%  TODO mention Koppen in analysis

\subsubsection{Potential future work}

\subsection{Implications for event attribution}\label{subsec:disfield}