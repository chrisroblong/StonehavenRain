\documentclass[12pt,a4paper]{report}

\usepackage{graphics}
\usepackage{fullpage,epsf,graphicx}%, amstext,url}
\graphicspath{{figures/}}
\def\BibTeX{{\rm B\kern-.05em{\sc i\kern-.025em b}\kern-.08em
    T\kern-.1667em\lower.7ex\hbox{E}\kern-.125emX}}

\begin{document}

\thispagestyle{empty}

%
%	This is a basic LaTeX Template for the TP/MP MSc Dissertation report

\parindent=0pt          %  Switch off indent of paragraphs 
\parskip=5pt            %  Put 5pt between each paragraph  

%	This section generates a title page
%       Edit only the sections indicated to put in the project title, and submission date

\vspace*{0.1\textheight}

\begin{center}
        \huge{\bfseries Event Attribution}\\ % Replace with the title of your dissertation!
\end{center}

\medskip

\begin{center}
        \Large{Christopher Long}\\  % Author of dissertation - replace with your name!
        \medskip
        \large{August 18, 2023}  % Submission date
\end{center}

%%% If necessary, reduce the number 0.4 below so the University Crest
%%% and the words below it fit on the page.
%%% Don't let the crest, or the wording below it, flow onto the next page!

\vspace*{0.4\textheight}

\begin{center}
        \includegraphics[width=35mm]{crest.pdf}
\end{center}

\medskip

\begin{center}

\large{
  MSc in Mathematical Physics\\[0.8ex]
  The University of Edinburgh\\[0.8ex]
  2023}

\end{center}

\newpage


\pagenumbering{roman}

\begin{abstract}
Extreme rainfall on Wednesday 12 August 2020 resulted in debris being washed out of a drain and onto the Dundee-Aberdeen line at Carmont, Aberdeenshire.

This caused the Stonehaven derailment,
    in which three people were killed and six people were injured,
    leaving no passengers on the train unharmed.

There is evidence that future change to the climate makes extreme weather events, including rainfall,
    more common.

\end{abstract}

\pagenumbering{roman}

\begin{center}
\textbf{Declaration}
\end{center}

I declare that this report is entirely my own work unless stated otherwise.

Chapter~\ref{ch:attribution} contains background material,
    with the first and third sections primarily relying on World Weather Attribution's methodology~\cite{van_Oldenborgh_et_al_2021},
    and the second section makes use of the information in Coles' textbook on the topic~\cite{Coles_2001}.

Chapter~\ref{ch:dev} consists of my own work,
    excepting the use of code and creation of equivalent diagrams to a currently approved but not yet published paper by my supervisor.
Code used in this project is given in~\cite{Me_Code},
    code used in my supervisor's paper is given in~\cite{Tett_Code}

Chapter~\ref{ch:results} is entirely my own work.

\bigskip

Calculations used Python with many standard packages.

NumPy and XArray are used to manage and process both real and modelled rainfall data.

CartoPy is used to process data between both Ordinance Survey and Longitude-Latitude coordinate systems.

R and the RPy package are used to fit extreme value distributions to the rainfall data.

The data processing ran on both the JASMIN supercomputer and on my local machine,
GitHub was used to manage the transfer of both code and data.

The majority of the code used in this project is based on the code used by my supervisor performing a similar analysis on the 2021 Edinburgh cloudburst.


\newpage

\begin{center}
\textbf{Personal Statement} % Use Git commits
\end{center}

I spent the first 2 weeks of the project retro-fitting the Edinburgh castle
library to apply to the Stonehaven crash.
This process began by creating plots including the
rainfall at the crash location on the day and an animation of the rainfall
in the Stonehaven region, allowing me to find an appropriate event definition.

For the remaining weeks of June, I implemented a topographical height mask
to the radar data, as well as establish a workspace to transfer data between
my laptop and the JASMIN supercomputer.

Following this I began my analysis of the radar data.
This involved computing and plotting the monthly rainfall maxima in
the Stonehaven region, as well as finding the quantiles of these extrema,
as well as using R to find the parameters of a GEV fit to both the empirical
and a monte carlo bootstrap of the rainfall distribution.

In mid-July, I created plots for these distributions and started to work with the Convection Permitting Model data,
    processing it as necessary to be appropriate for the use case.
This allowed me to create the time series to act as covariates to the CPM rainfall data,
    which were used in the first week of August to compute the final outcomes and plots.


I started writing this dissertation in mid-July, and I spent the first
three weeks of August working on it full-time.

I have achieved x, y, z results, but understand that my findings are limited.
With additional time,
    I could then consider the effects of different event definition,
    other than the one-hour rainfall within the grid areas given,
    as well as used different models and quantified the difference in these approaches.
This is discussed further in section~\ref{sec:discussion}.


\newpage

\begin{center}
%\vspace*{2in}
% an acknowledgements section is completely optional but if you decide
% not to include it you should still include an empty {titlepage}
% environment as this initialises things like section and page numbering.
\textbf{Acknowledgements}
\end{center}

I'd like to thank my supervisor Professor Simon Tett for
making this project possible.



\tableofcontents
\listoftables
\listoffigures

\pagenumbering{arabic}

\chapter{Introduction}\label{ch:intro}
The Introduction should contain a description of your project and the
problem you are trying to solve. It should start off at a level that
should be understandable by anyone with a degree in physics, but it
can become more technical later

Where appropriate you should include references to work that has
already been done on your topic and anything else which lets you set
your work in context.


\chapter{Event Attribution Theory}\label{ch:attribution}

As a part of a complex system,
    any weather event is necessarily multifactorial,
    with no single cause.
It is therefore necessary to use a probabilistic lens to perform a scientific analysis.

In this case, the weather at a given place and time of year is expressed as a statistical distribution of events.
The climate may then be considered either the distribution itself,
    or the parameters informing the distribution.
In other words, ``Climate is what we expect, weather is what we get.''~\cite{Herbertson_1935}.

The scientific method may then be applied,
    using the climate distribution as a dependent variable to the factor being considered.
Issues, of both a technical and a theoretical nature can arise with this approach,
    and will be discussed in subsection~\ref{subsec:attrchallenge}.

In all the examples given in this section,
    the independent variable is anthropogenic climate change.
This is often quantified by using a temperature metric,
    although additional factors may be considered.

It may be noted that,
    due to their generality,
    climate event attribution techniques may be of use outside risk analysis from modern changes in the climate.
% TODO: add about historical extremes Ruddiman

An event attribution study aims to provide two quantities.
The first is the change in the probability of the event.
The second is the change in the magnitude of the event.

\subsubsection{Development of Event Attribution}

In 2003, whilst experiencing flooding as a result of extreme weather,
    Allen~\cite{Allen_2003} posited the probabilistic interpretation of climate change impacts,
    proposing a potential to find greenhouse gas emitters liable for extreme weather events.
%  Add note about the fact that a grant would not be paid to emitters for reducing the likelihood.
A notable early use case of this probabilistic approach was by Stott et al.~\cite{Stott_2004},
    finding that 75\% of the risk of the European heatwave is attributable to anthropogenic climate change.

World Weather Attribution produces rapid Event Attribution studies,
    some of which are within a month of the event occuring~\cite{van_Oldenborgh_et_al_2018},
    using the probabilistic approach.
Alternative approaches also exist,
    such as the `storyline' approach,
    outlined by Shepherd et al. in~\cite{Shepherd_et_al_2018},
    although these may be considered contentious~\cite{García-Portela_Maraun_2023}.

\section{Steps to Event Attribution}\label{subsec:attrsteps}

The precise number of steps for successful event attribution vary by the situation.
Van Oldenborgh et al.~\cite{van_Oldenborgh_et_al_2021} give 8 separate steps,
    while Otto~\cite{Otto_2017} gives 6 steps.
Both of these methodologies are functionally identical;
    I shall base my description upon 4 of the steps described by Otto.

Only four steps are given in full below.
This is due to the removal of an analysis trigger step at the start of the process
    and a communication step at the end of the process,
    both of which I will cover here briefly.

An analysis trigger is a set of criteria used by researchers to determine whether an event should be investigated.
These serve a utilitarian purpose,
    allowing the events with the greatest impact to be investigated first.
These may include the number of deaths or number impacted,
    as in chapter 2 of~\cite{van_Oldenborgh_et_al_2021},
    or economic and cultural factors, as in~\cite{Tett_Soon}.
The former criteria may skew analysis towards poorer nations~\cite{Kahn_2005},
    while the latter may skew analysis towards richer nations.

It is of note that potential liability of anthropogenic climate change may be a triggering factor,
    so the analysis begins due to a hypothesis that the event was caused by anthropogenic climate change.
I believe that this may bias the results of a meta-analysis of event attribution studies,
    and so increase the challenge in determining anthropogenic climate change as a cause of more extreme events in general.
I omit giving any further detail on the analysis trigger as it is beyond the scope of mathematical or physical interest.

The communication of the outcome of the analysis may also be considered in the study.
This is especially true outside the scientific community
    and considers the risk of the audience drawing incorrect conclusions,
    as well as the risk of policymakers acting on incorrect information.


\subsection{Event Definition}\label{subsec:backeventdef}

To perform an attribution analysis,
    the weather event must first be defined in clear and quantitative terms.
This requires specifying a variable with a threshold that has been crossed,
    as well as the time and region at which an event may be considered.

This is not necessarily straightforward.
An event attribution is normally triggered by the event's social or environmental impact,
     not just due to the meteorological significance.
Therefore, care must be taken to ensure that the definition of the event is sufficiently similar to the conditions that would cause a similar impact.

Many of the challenges in Event Attribution arise directly from a poor Event Definition.
Events that are too narrowly defined may have issues with reproducibility,
    as will be discussed in~\ref{subsec:eventrepro},
    while events that are too widely defined may give invalid results,
    as will be discussed in~\ref{subsec:modelvalid}.

\subsubsection{Variable and Threshold}

The climate variable used to define the event requires travelling up the causal chain that leads to it.
As a set of examples,
    floods/droughts are caused by extreme highs/lows in rainfall,
    while heat waves and cold spells are caused by extreme high or low temperatures.

However,
    all of these events require the extreme rainfall/temperature to be sustained for the event to be hazardous.
Simple metrics may be applicable here,
    as the definition can then use the rainfall/temperature averaged over a given time period.
The selection of time period itself also requires a consideration of the event.
A damaging heavy storm requires extreme winds over a matter of minutes,
    while it may take days for the land to dry out sufficiently for a drought to occur.

More advanced metrics may be used.
For example, an Expert Team on Climate Change Detection Indices~\cite{Zhang_2011}
    describes a `Growing Season Length',
    allowing climate-caused famine to be quantified using only daily temperature data.
A `Wet-Bulb Temperature' has been used to combine humidity and temperature data to create an index that better measures teh effect of a heatwave on human health~\cite{Li_2020}.

\subsubsection{Spatial and Temporal Region}

The area
%  TODO El Nino, Season, Region, vary variable within region or not, homogenous weather

\subsection{Model Evaluation}\label{subsec:backmodeleval}

\subsection{Likelihood Estimation}\label{subsec:backlikeest}

\subsection{Interpretation}\label{subsec:backinterp}

\subsection*{Alternative Approaches}\label{subsec:altappro}

\section{Extreme Value Statistics}\label{sec:exstats}

\subsection{GEV Distribution}\label{subsec:gev}

\subsection{Generalised Pareto Distribution}\label{subsec:gpd}

\section{Challenges in Event Attribution}\label{sec:attrchallenge}

\subsection{Event Reproducibility}\label{subsec:eventrepro}

\subsection{Statistical Independence}\label{subsec:statindep}

\subsection{Validity of Model}\label{subsec:modelvalid}

\chapter{Development}\label{ch:dev}


This section should be written in standard scientific
language. Standard techniques in your research field should not be
written out in detail. In computational projects this section should
be used to explain the algorithms used and the layout of the
computational code. A copy of the actual code may be given in the
appendices if appropriate.

This section should emphasise the philosophy of the approach used and
detail novel techniques. However please note: this section should not
be a blow-by-blow account of what you did throughout the project. It
should not contain large detailed sections about things you tried and
found to be completely wrong! However, if you find that a technique
that was expected to work failed, that is a valid result and should be
included.

Here logical structure is particularly important, and you may find
that to maintain good structure you may have to present the
explorations/calculations/computations/whatever in a different order
from the one in which you carried them out.

\section{Event Definition}\label{sec:def}

\section{Event Modelling}\label{sec:model}

\chapter{Results and Analysis}\label{ch:results}

This section should detail the obtained results in a clear,
easy-to-follow manner. It is important to make clear what are original
results and what are repeats of previous calculations or computations.
Remember that long tables of numbers are just as boring to read as
they are to type-in!

Use graphs to present your results wherever practicable.

Results or computations should be presented with uncertainties
(errors), both statistical and systematic where applicable.

Be selective in what you include: half a dozen \emph{e.g.}~tables that
contain wrong data you collected while you forgot to switch on the
computer are not relevant and may mask the correct results.

\section{Discussion of your results}\label{sec:discussion}

This section should give a picture of what you have taken out of your
project and how you can put it into context.

This section should summarise the results obtained, detail conclusions
reached, suggest future work, and changes that you would make if you
repeated the project.

\chapter{Conclusions}\label{ch:conclusions}

This is the place to put your conclusions about your work. You can
split it into different sections if appropriate. You may want to include
a section of future work which could be carried out to continue your
research.

The conclusion section should be at least one page long, preferably 2
pages, but not much longer.

\appendix
% the appendix command just changes heading styles for appendices.

\chapter{Stuff that's too detailed}

Appendices should contain all the material which is considered too
detailed to be included in the main body of the text, but which is
important enough to be included in the thesis.

Perhaps this is a good place to mention \BibTeX.

You can do references in the simple way explained in the introduction,
or you can use \BibTeX.


\bibliographystyle{unsrt}
\bibliography{ref}




\end{document}

