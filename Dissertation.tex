\documentclass[12pt,a4paper]{report}

\usepackage{graphics}
\usepackage{fullpage,epsf,graphicx}
\usepackage{amsmath}%, amstext,url}
\usepackage{comment}
\usepackage{hyperref}
\usepackage{verbatim}
\usepackage{float}
\usepackage{placeins}
\usepackage{subcaption}
\graphicspath{{figures/}}
\def\BibTeX{{\rm B\kern-.05em{\sc i\kern-.025em b}\kern-.08em
    T\kern-.1667em\lower.7ex\hbox{E}\kern-.125emX}}

\begin{document}
%  TODO copy paste into google doc and fix grammar and spelling.
\thispagestyle{empty}

%
%	This is a basic LaTeX Template for the TP/MP MSc Dissertation report

\parindent=0pt          %  Switch off indent of paragraphs 
\parskip=5pt            %  Put 5pt between each paragraph  

%	This section generates a title page
%       Edit only the sections indicated to put in the project title, and submission date

\vspace*{0.1\textheight}

\begin{center}
        \huge{\bfseries Event Attribution}\\ % Replace with the title of your dissertation!
\end{center}

\medskip

\begin{center}
        \Large{Christopher Long}\\  % Author of dissertation - replace with your name!
        \medskip
        \large{August 18, 2023}  % Submission date
\end{center}

%%% If necessary, reduce the number 0.4 below so the University Crest
%%% and the words below it fit on the page.
%%% Don't let the crest, or the wording below it, flow onto the next page!

\vspace*{0.4\textheight}

\begin{center}
        \includegraphics[width=35mm]{crest}
\end{center}

\medskip

\begin{center}

\large{
  MSc in Mathematical Physics\\[0.8ex]
  The University of Edinburgh\\[0.8ex]
  2023}

\end{center}

\newpage


\pagenumbering{roman}

\begin{abstract}

Extreme rainfall on 12 August 2020 caused the Stonehaven Derailment,
    which resulted in three deaths.
This project applies current event attribution techniques for extreme rainfall events,
    finding probability ratios for the extreme rainfall to be 1.10 (1.06--1.15 90\% CI),
    1.37 (1.22--1.61) and 1.78 (1.46--2.34) for the 1980s, 2010s and in a 2K warmer world
    respectively over pre-industrial.

\end{abstract}

\pagenumbering{roman}

\begin{center}
\textbf{Declaration}
\end{center}

I declare that this report is entirely my own work unless stated otherwise.

The introduction, chapter~\ref{ch:intro}, describes the motivations for the work.
This includes the information showing the impact of the extreme rainfall on the derailment by the Rail Accident Investigation Board~\cite{RAIB_2022},
    as well as my supervisor's paper analysing the effect of climate change on the risk of an extreme rainfall event~\cite{Tett_Soon}.

Chapter~\ref{ch:attribution} contains background material,
    with the first and third sections primarily relying on World Weather Attribution's methodology~\cite{van_Oldenborgh_et_al_2021},
    and the second section makes use of the information in Coles' textbook on the topic~\cite{Coles_2001}.

Chapter~\ref{ch:dev} consists of my own work,
    excepting the use of code and creation of equivalent diagrams to a currently approved but not yet published paper by my supervisor.
Code used in this project is given in~\cite{Me_Code},
    code used in my supervisor's paper is given in~\cite{Tett_Code}

Chapter~\ref{ch:results} is entirely my own work,
    using the same plotting techniques as in~\cite{Tett_Soon}.

\bigskip

Calculations used Python with many standard packages.

NumPy and XArray are used to manage and process both real and modelled rainfall data.

CartoPy is used to process data between both Ordinance Survey and Longitude-Latitude coordinate systems.

R and the RPy package are used to fit extreme value distributions to the rainfall data.
This calls the extRemes R package~\cite{extremes_R}.

The Rain Radar Data is taken from the Met Office's NIMROD system~\cite{radar_data}.

The Model Data is taken from the Met Office Hadley Centre's UKCP18 Convection-Permitting Model projections~\cite{model_data}.

Tht topography data on an OSGB grid is taken from~\cite{radar_topog}.

The data processing ran on both the JASMIN supercomputer and on my local machine,
GitHub was used to manage the transfer of both code and data.

The majority of the code used in this project is based on the code used by my supervisor performing a similar analysis on the 2021 Edinburgh cloudburst~\cite{Tett_Soon}.


\newpage

\begin{center}
\textbf{Personal Statement} % Use Git commits
\end{center}

I spent the first 2 weeks of the project retro-fitting the Edinburgh castle
library to apply to the Stonehaven crash.
This process began by creating plots including the
rainfall at the crash location on the day and an animation of the rainfall
in the Stonehaven region, allowing me to find an appropriate event definition.

For the remaining weeks of June, I implemented a topographical height mask
to the radar data, as well as establish a workspace to transfer data between
my laptop and the JASMIN supercomputer.

Following this I began my analysis of the radar data.
This involved computing and plotting the monthly rainfall maxima in
the Stonehaven region, as well as finding the quantiles of these extrema,
as well as using R to find the parameters of a GEV fit to both the empirical
and a monte carlo bootstrap of the rainfall distribution.

In mid-July, I created plots for these distributions and started to work with the Convection Permitting Model data,
    processing it as necessary to be appropriate for the use case.
This allowed me to create the time series to act as covariates to the CPM rainfall data,
    which were used in the first week of August to compute the probability ratios, intensity ratios and plots.
Any work done in August other than writing up the dissertation was spent analysing and critiquing the techniques used.

I started writing this dissertation in mid-July, and I spent the first
three weeks of August working on it full-time.

I have achieved the final outcome of probability ratios for the event and
    discussed the limitations of these results.


\newpage

\begin{center}
%\vspace*{2in}
% an acknowledgements section is completely optional but if you decide
% not to include it you should still include an empty {titlepage}
% environment as this initialises things like section and page numbering.
\textbf{Acknowledgements}
\end{center}

I'd like to thank my supervisor Professor Simon Tett for
making this project possible.



\tableofcontents
\listoftables
\listoffigures

\pagenumbering{arabic}

\chapter{Introduction}\label{ch:intro}

\section{Stonehaven Derailment}\label{sec:stonederail}

%  TODO use RAIB analysis

\section{Rainfall in a warmer world}\label{sec:warmerrainfall}

%  TODO rely on Simon's paper
%  TODO rely on IPCC report

\begin{comment}
The Introduction should contain a description of your project and the
problem you are trying to solve. It should start off at a level that
should be understandable by anyone with a degree in physics, but it
can become more technical later

Where appropriate you should include references to work that has
already been done on your topic and anything else which lets you set
your work in context.
\end{comment}

\chapter{Event Attribution Theory}\label{ch:attribution}
As a part of a complex system,
    any weather event is necessarily multifactorial,
    with no single cause.
It is therefore necessary to use a probabilistic lens to perform a scientific analysis.

In this case, the weather at a given place and time of year is expressed as a statistical distribution of events.
The climate may then be considered either the distribution itself,
    or the parameters informing the distribution.
In other words, ``Climate is what we expect, weather is what we get.''~\cite{Herbertson_1935}.

The scientific method may then be applied,
    using the climate distribution as a dependent variable to the factor being considered.
Issues, of both a technical and a theoretical nature can arise with this approach,
    and will be discussed where they may arise in the process.

In all the examples given in this section,
    the independent variable is anthropogenic climate change.
This is often quantified by using a temperature metric,
    although additional factors may be considered.

It may be noted that
    climate event attribution techniques may be of use outside risk analysis from modern changes in the climate,
    and generalised to cover large-scale events due to other factors changing in the climate.
For example,
    Ruddiman~\cite{Ruddiman_2010} has attempted to attribute the lack of a recent glaciation to the beginning of agriculture.

An event attribution study aims to provide two quantities.
The first is the change in the probability of the event.
The second is the change in the intensity of the event.

\subsubsection{Development of Event Attribution}

In 2003, whilst experiencing flooding as a result of extreme weather,
    Allen~\cite{Allen_2003} posited the probabilistic interpretation of climate change impacts,
    proposing a potential to find greenhouse gas emitters liable for extreme weather events.
%  Add note about the fact that a grant would not be paid to emitters for reducing the likelihood.
A notable early use case of this probabilistic approach was by Stott et al.~\cite{Stott_2004},
    finding that 75\% of the risk of the European heatwave is attributable to anthropogenic climate change.

World Weather Attribution produces rapid Event Attribution studies,
    some of which are within a month of the event occuring~\cite{van_Oldenborgh_et_al_2018},
    using the probabilistic approach.
Alternative approaches also exist.
One of these is the `Boulder' methodology,
    described in section 2.2 of~\cite{Otto_2017},
    which seeks to break down the proportions to which the event was caused by natural variability and the due to Anthropogenic Climate Change.
Another is the `storyline' methodology,
    outlined by Shepherd et al.\ in~\cite{Shepherd_et_al_2018},
    although this has been considered contentious~\cite{García-Portela_Maraun_2023}.

\section{Steps to Event Attribution}\label{sec:attrsteps}

The precise number of steps for successful event attribution vary by the situation.
Van Oldenborgh et al.~\cite{van_Oldenborgh_et_al_2021} give 8 separate steps,
    while Otto~\cite{Otto_2017} gives 6 steps.
Both of these methodologies are functionally identical;
    I shall base my description upon 4 of the steps described by Otto.

Only four steps are given in full below.
This is due to the removal of an analysis trigger step at the start of the process
    and a communication step at the end of the process,
    both of which I will cover here briefly.

An analysis trigger is a set of criteria used to determine whether an event should be investigated.
These serve a utilitarian purpose,
    allowing the events with the greatest impact to be investigated first.
These may include the number of deaths or number impacted,
    as in chapter 2 of~\cite{van_Oldenborgh_et_al_2021},
    or economic and cultural factors, as in~\cite{Tett_Soon}.
The former criteria may skew analysis towards poorer nations~\cite{Kahn_2005},
    while the latter may skew analysis towards richer nations.

It is of note that potential liability of anthropogenic climate change may be a triggering factor,
    so the analysis begins due to a hypothesis that the event was caused by anthropogenic climate change.
I believe that this may bias the results of a meta-analysis of event attribution studies,
    and so increase the challenge in determining anthropogenic climate change as a cause of more extreme events in general.
I omit giving any further detail on the analysis trigger as it is beyond the scope of mathematical or physical interest.

The communication of the outcome of the analysis may also be considered in the study.
This is especially true outside the scientific community
    and considers the risk of the audience drawing incorrect conclusions,
    as well as the risk of policymakers acting on incorrect information.

\subsection{Event Definition}\label{subsec:backeventdef}

To perform an attribution analysis,
    the weather event must first be defined in clear and quantitative terms.
This requires specifying a variable with a threshold that has been crossed,
    as well as the time and region at which an event may be considered.

This is not necessarily straightforward.
An event attribution is normally triggered by the event's social or environmental impact,
     not just due to the meteorological significance.
Therefore, care must be taken to ensure that the definition of the event is sufficiently similar to the conditions that would cause a similar impact.

Many of the challenges in Event Attribution arise directly from a poor Event Definition.
Events that are too narrowly defined not be reproducible by climate models
    while events that are too widely will give results for more minor events that would not have had the same impacts as the actual event.

\subsubsection{Variable and Threshold}

The climate variable used to define the event requires travelling up the causal chain that leads to it.
As a set of examples,
    floods/droughts are caused by extreme highs/lows in rainfall,
    while heat waves and cold spells are caused by extreme high or low temperatures.

However,
    all of these events require the extreme rainfall/temperature to be sustained for the event to be hazardous.
Simple metrics may be applicable here,
    as the definition can then use the rainfall/temperature averaged over a given time period.
The selection of time period itself also requires a consideration of the event.
A damaging heavy storm requires extreme winds over a matter of minutes,
    while it may take days for the land to dry out sufficiently for a drought to occur.

More advanced metrics may be used.
For example, an Expert Team on Climate Change Detection Indices~\cite{Zhang_2011}
    describes a `Growing Season Length',
    allowing climate-caused famine to be quantified using only daily temperature data.
A `Wet-Bulb Temperature' has been used to combine humidity and temperature data to create an index that better measures teh effect of a heatwave on human health~\cite{Li_2020}.

\subsubsection{Spatial and Temporal Region}

In addition to defining a threshold variable,
    it is necessary to define the regions and times at which an event can be considered sufficiently similar to that being analysed.
This is due to the fact that both historical trend analysis is impossible on a single extreme event at a single place on a single day of the year
    and that climate models will rarely reproduce events in such a narrow definition.

For the spatial region,
    a region with a known climate similar to that in which the event occurred is chosen,
    typically directly around the location of the event.
This is done so that the frequency and types of event will match the location of the original event,
    as well as to ensure that the factors that change the climate,
    whether Anthropomorphic or not,
    are similar to those changing the climate at the point of the event.

In addition to the aforementioned similarity in location,
    a historical trend analysis can aid in defining a region,
    along with tools including topography and climate classifications.
The region is chosen based on similar weather,
    not a similarity in hazard risk,
    as the goal is to perform an attribution for the event,
    not to advise on potential future risk.
For example,
    a region defined for a flood event over a flood-prone area would only consider areas with similar rainfall,
    not those similarly flood-prone.

For the temporal region,
    or time period for the event to be considered,
    an analogous process must be used.
It is often appropriate to use a seasonal time period,
    as this can also limit the event to those with similar meteorology,
    such as by only considering monsoon rainfall~\cite{Otto_et_al_2023}.
This can also prove to be a practical step,
    reducing the amount of data to be used later in the attribution process.

Additional `Boundary Conditions'~\cite{van_Oldenborgh_et_al_2021} may then also be imposed.
Of note is the importance of the El Niño,
    especially for drought events in the southern hemisphere~\cite{Lyon_2004},
    so this may also be accounted for when defining these event.

Once all of these definitions have been made,
    a `class' of events has been described.
This class will then be used to find like events in both the historical and model data.

\subsection{Model Evaluation}\label{subsec:backmodeleval}

%  TODO Model Validity and Reproducability

\subsection{Likelihood Estimation}\label{subsec:backlikeest}

%  TODO Statistical Independence

\subsection{Interpretation}\label{subsec:backinterp}

\section{Extreme Value Statistics}\label{sec:exstats}

This subsection will describe the statistical foundations of extreme event modelling,
    paraphrasing Coles' book~\cite{Coles_2001} unless where otherwise stated.
Climate events that pose hazards are necessarily extreme.
Therefore, to model these events,
    extreme value distributions are necessary.

Throughout the analysis,
    we assume that the variables are independent and identically distributed ($i.i.d.$).
%  TODO Statistical independence

For event definitions which average a variable over a season,
    as opposed to focusing on the extreme
    a normal distribution is instead appropriate.
This is as the Central Limit Theorem states that the limit of the mean of $n$ $i.i.d.$ random variables tends to a normal distribution.
The equivalents for extreme events, the GEV distribution and Extremal Types Theorem,
    will be explained in due course.

\subsection{GEV Distribution}\label{subsec:gev}

\subsubsection{Extremal Types Theorem}

This explanation is as given in section 3.1 of~\cite{Coles_2001}.

Let $M_n$ be the maximum of $n$ $i.i.d.$ random variables.
If there exist sequences of constants $a_n > 0$ and $b_n$, and non-degenerate $G$ such that:

\[ P\left( \frac{M_n - b_n}{a_n}  \leq z \right) \rightarrow G(z) \text{ as } n \rightarrow \infty \]

Then $G$ is a member of the GEV (Generalised Extreme Value) family.

Section 3.1.4  of~\cite{Coles_2001} outlines a proof of the Extremal Types Theorem.
This involves showing that being a GEV distribution is equivalent to being `Max-Stable',
    then that the maxima of GEV distributions are Max-Stable,
    giving the GEV as the distribution of the original maximum in the infinite limit.
It is of note that this is similar to one approach to proving the Central Limit Theorem,
    which considers the normal distribution as a unique fixed point under convolutions,
    resulting in normal distributions being an infinite limit of the sum or mean~\cite{Hamedani_Walter_1984}.

Section 3.5 of~\cite{Coles_2001} shows, \textit{mutatis mutandis},
    that the extremal types theorem applies identically to minima by reversing the sign of the data.

\subsubsection{Parameters and Types of GEV distributions}

The GEV distribution has a CDF given by:
\begin{equation}\label{eq:gevcdf}
    G(z) = \exp \left( - \left( 1 + \xi \left( \frac{z-\mu}{\sigma} \right)  \right)^{-\frac{1}{\xi}} \right)
\end{equation}

This has three parameters, the location $\mu$, the scale $\sigma$ and the shape $\xi$.
It is clear from this CDF that the change of location and scale parameters represent moving and rescaling the distribution,
    much as the mean and standard deviation parameters do for a normal distribution.

$1-G(z)$ is the survival function of the distribution.
This function gives the probability of an interval containing an event more extreme (of larger intensity) than $z$.

The shape parameter determines the thickness of the tail of the distribution.
The larger the shape parameter, the thicker the tail of the distribution.
This is visible in figure~\ref{fig:gevshape}.

Where $\xi > 0$,
    the distribution is heavy-tailed.
This type of GEV distribution is also known as a Fréchet distribution.
Special cases of this distribution have unique properties,
    where $\xi \geq \frac{1}{2}$, the distribution has infinite variance and
    where $\xi \geq 1$, the distribution has infinite mean.

Where $\xi < 0$, the distribution is bounded,
    also known as a reversed Weibull distribution.
These distributions are limited to a maximum `most extreme' value of $\mu - \frac{\sigma}{\xi}$.
It can therefore be expected that distributions modelling extrema with physical constraints would have a negative shape parameter.
For example, Table 3 of Chikobvu and Chifurira's paper~\cite{Chikobvu_2015} modelling rainfall minima finds a negative shape parameter,
    which would be expected as rainfall cannot be negative.

Where $\xi = 0$,
    the distribution is a Gumbel distribution.
The CDF of this distribution requires taking the limit as $\xi \rightarrow 0$,
    giving a double exponential CDF\@.
The extrema of normal distributions converge to a Gumbel distribution,
    shown as a step in the proof of Proposition 1 of~\cite{Bailey_2014}.
This means that, with the Central Limit Theorem,
    the maxima of a sample of means of any distribution approaches a Gumbel distribution as both samples grow larger,
    provided the samples are independent.

The above paragraph suggests that a Gumbel distribution is appropriate for modelling rainfall maxima,
    as the rainfall is averaged over a time period and then a maxima is taken over those time periods for a given year or season.
However, this has not been found to be true.
Koutsoyiannis~\cite{Koutsoyiannis_2003} finds both an ``extremely slow'' theoretical convergence to a Gumbel distribution and
    that the Gumbel distribution does not fit empirical rainfall maxima.
Therefore, the three-parameter GEV distribution is used to model rainfall maxima in event attribution.

\subsubsection{Return levels of GEV distributions}

Equation~\ref{eq:gevcdf} can be inverted to give a return level $z_p$ for a probability $p$, given in~\cite{Coles_2001}:
\begin{equation}\label{eq:gevreturn}
    z_p = \mu - \frac{\sigma}{\xi}\left( 1-\left( -\log\left( 1-p \right) \right)^{-\xi} \right)
\end{equation}
This is otherwise known as the inverse survival function,
    where $z_p$ is the value for which the probability of finding a value greater than $z_p$ is $p$.
The critical value for a given probability $p$ breaks down into the sum of two components,
    the location and a quantity that is proportionate to the scale.

The rarity of events for a return period $P$ can be computed,
    as the probability $p$ is equal to $1/P$.
This is allows the GEV distribution to use the language common for extreme weather events,
    as with knowledge of the underlying distribution,
    it is possible to compute the intensity of an event for any return period.

\subsection{GEV Parameter Estimation}\label{subsec:parameterest}

In sections~\ref{subsec:radardatafit} and~\ref{subsec:radardatafit},
    parameters are fitted to data using the extRemes R library~\cite{extremes_R}.
This software uses MLE (Maximum Likelihood Estimation) to get best estimates of the parameters of a dataset,
    assuming the data obeys a GEV distribution.
This technique will be covered briefly here.

The extRemes package is also used to fit linear changes in a parameter with respect to a covariate.
Although this uses the same underlying mathematics as this section,
    relying on MLE,
    the exact algorithm used is beyond the scope of this report.

\subsubsection{Maximum Likelihood Estimation}
%  TODO

\subsubsection{Akaike Information Criterion}

To assess the goodness of fit,
    the Akaike Information Criterion (AIC)~\cite{AIC_1974} is used.
This is given in the following equation:
\begin{equation}\label{eq:AIC}
    AIC = 2k - 2\ln \left( L \right)
\end{equation}
Where $k$ is the number of parameters and $L$ is the value of the likelihood function with the data.

A better fit has a lower AIC .
Equation~\ref{eq:AIC} imposes a cost of 2 per parameter,
    preventing overfit models with many parameters being considered a good fit for the data.

For Models 1 and 2, the probability that model 2 loses less information than model 1,
    i.e. that model 2 is a better fit than model 1,
    is given in~\cite{AIC_Info}:
\begin{equation}\label{eq:AIC_Info}
    \exp \left( \frac{1}{2} \left( AIC_1 - AIC_2 \right) \right)
\end{equation}


\chapter{Development}\label{ch:dev}

Throughout this chapter,
    reference is made to the `Stonehaven Region'.
This is defined as +/- 50km of the crash location for radar data,
    which uses an ordinance survey grid,
    and as +/- 0.5 degrees of the crash location for the radar data.

\section{Empirical Data}\label{sec:def}

The data used in this section is taken from the Met Office's NIMROD system~\cite{radar_data}.
This data has a 5km by 5km spatial resolution and a 15-minute temporal resolution,
    although data is also available in a 1km by 1km spatial resolution.
The coarser dataset was chosen as the Stonehaven event was a Summer convective storm,
    which occurs on scales great enough to be captured at a 5km resolution.

\subsection{The Stonehaven Event}\label{subsec:actualevent}

\subsubsection{Rainfall at the crash location}

\begin{figure}[H]
    \begin{center}
    \includegraphics[width=100mm]{stonehavendayraingraph}
    \end{center}
    \caption[A graph of the rainfall at the Stonehaven crash location.]{
        A graph of the rainfall at the Stonehaven crash location.
    X-axis is the date and hour (24-hour clock) of the rainfall measurement,
        ranging across the morning of the 12th of August 2020.
    Y-axis is the rainfall measured in mm/h.}
    \label{fig:stonehavendayraingraph}
\end{figure}

From figure~\ref{fig:stonehavendayraingraph},
    the rainfall event begins at 4:15 and ends at 8:15.
The rainfall at the crash location is then given in sixteen 15-minute chunks of time,
    or as across four hours.

This may suggest a four-hour event definition should be used.
However, this was not done for two reasons.
First, it is clear that the rainfall occurred in two 2-hour peaks,
 so a 4-hour definition would not be appropriate for this event.
Second, in section~\ref{subsec:radarprocess},
    the radar data is processed into chunks,
    each of which start on an increment of the length of time of the event.
While this would accurately capture this 4-hour storm, being between 4:00 and 8:00,
    an event from 6:00 to 10:00 would be considered two events of half the size.

A one-hour peak was then used instead.
This was as it allowed storms of a differing lengths to be quantified by their peak rainfall.

\subsubsection{Rainfall in the Stonehaven region}

\begin{figure}[H]
    \centering

    \begin{subfigure}{0.48\textwidth}
        \centering
        \includegraphics[width=\linewidth]{stonerain5}
    \end{subfigure}
    \hfill
    \begin{subfigure}{0.48\textwidth}
        \centering
        \includegraphics[width=\linewidth]{stonerain6}
    \end{subfigure}

    \vspace{\baselineskip}

    \begin{subfigure}{0.48\textwidth}
        \centering
        \includegraphics[width=\linewidth]{stonerain7}
    \end{subfigure}
    \hfill
    \begin{subfigure}{0.48\textwidth}
        \centering
        \includegraphics[width=\linewidth]{stonerain8}
    \end{subfigure}
    \caption[Map of rainfall in the Stonehaven region at times in the morning of 12th August 2020.]{
        Map of rainfall in the Stonehaven region (+/- 50km of Stonehaven)
        at times of 5, 6, 7 and 8AM of the 12th August 2020,
        in the top left, top right, bottom left and bottom right respectively.
    Scale is mm/h.
    Black lines are coastlines, red lines are local authority boundaries.
    Purple dot is the crash location.}
    \label{fig:stoneregionrain}
\end{figure}

Figure~\ref{fig:stoneregionrain} shows the rainfall in the Stonehaven region,
    demonstrating the movement of the storm North-East throughout the morning of the event.
This figure is also available as an animation using data from 15-minute intervals,
    showing the storm moving as expected.

The figure also suggests that the event covers a large area,
    validating the use of the coarser-resolution 5km dataset as sufficient to describe the event.
One unfortunate consequence of using the 5km dataset is on the rainfall described in figure~\ref{fig:stonehavendayraingraph},
    as the crash location lies near the boundary of four cells, yet the data is only taken from the nearest cell,
    leading to the graph being less accurate approximation of the rainfall at the crash location than would be given by a 1km resolution dataset.

\subsection{Further Constraints}\label{subsec:furthercons}

\subsubsection{Stonehaven Geography}

\begin{figure}[H]
    \centering
    \begin{subfigure}{0.48\textwidth}
        \centering
        \includegraphics[width=\linewidth]{stonetopog90}
    \end{subfigure}
    \hfill
    \begin{subfigure}{0.48\textwidth}
        \centering
        \includegraphics[width=\linewidth]{stonetopog4950}
    \end{subfigure}
    \caption[Plot of the topography in the Stonehaven region.]{
        Plot of the topography in the Stonehaven region.
    Scale is in meters.
    Left is the raw data (at 90mx90m resolution),
    right is the resampled data (at 4950mx4950m resolution) as used for future processing.
    Black lines are coastlines, red lines are local authority boundaries.
    Red dot is the crash location.}
    \label{fig:stonetopog}
\end{figure}

The 4950m resolution data shown in figure~\ref{fig:stonetopog} was found by taking a mean average of 55 of the data points in each dimension,
    giving each grid cell an average of 3250 raw data points.

The area of the Stonehaven crash is around the 200m contour,
    with the crash occurring at around 100m,
    due to the cutting the railway passes through.
Looking at figure 26 of~\cite{RAIB_2022},
    the drain responsible for the crash is 150m above sea level.
Furthermore, the drain's catchment reaches close to 200m.

The topography of the region for the event attribution analysis was then chosen to be from 0m (sea level),
    to 400m.
This is done to make the topography around the crash location typical for the region in which the analysis is performed.
A 200m limit was suggested,
    to match the limit taken in~\cite{Tett_Soon},
    but was not taken forward as this had the potential to give too few data points to define an event.

\begin{figure}[H]
    \begin{center}
        \includegraphics[width=100mm]{stonetopogcut}
    \end{center}
    \caption[Map of the Stonehaven region with radar data grid cells within a range of 0--400m.]{
        Map of the Stonehaven region with radar data grid cells (5kmx5km) within a topographical range of 0--400m in yellow,
    grid cells outside of range in violet.
    Black lines are coastlines, red lines are local authority boundaries.
    Red dot is the crash location.}
    \label{fig:stonetopogallowed}
\end{figure}

Out of the 400 grid cells in the Stonehaven Region,
    data is taken from 185 of the grid cells.
This is as 194 of the grid cells are not over land,
    and an additional 21 grid cells have too high topography.

As the radar data is at a 5km resolution,
    while the resampled topography data is at a 4950m resolution,
    the XArray \texttt{interp\_like} method is used to get a linear interpolation for the height at each 5km grid cell.

\subsection{Processing and Fitting the Radar Data}\label{subsec:radarprocess}

The steps used in this section are identical to those used to processing and fitting the radar data for the Edinburgh cloudburst event~\cite{Tett_Soon},
    with changes made to fit the different definition and region in this case.

\subsubsection{Computing Maxima}

As the goal was to use an Extreme Value distribution to model the data,
    it is necessary to draw the extrema out from the data.
The dataset itself~\cite{radar_data} is composed of \texttt{.tar} files, each of which containing the radar data of a given day.

The \texttt{process\_radar\_data} Python script finds the maximum one-hour rainfall in each of these daily datasets and
    uses these to generate a dataset with the one-hour maximum rainfall and the time of the one-hour maximum rainfall
    at all points in the Stonehaven Region for a given month, in a NetCDF format
The \texttt{combine\_summary} Python script then takes these monthly maxima and their times and combines them into a single dataset,
    again as a NetCDF .

These Python scripts are called by the Shell scripts \texttt{run\_jobs} and \texttt{run\_combine} respectively.
The Shell scripts are run on a JASMIN Science Analysis Server and submit the Python scripts as SLURM jobs to the LOTUS Batch Processing Cluster,
    allowing all the monthly maxima to be computed in a single script and the combination of the datasets to access an amount of memory to make the task feasible.

The \texttt{get\_radar\_data} function in the \texttt{stonehavenRainLib} Python script then selects the summer months and
    combines them to get the maximum hourly rainfall in the summer of each year,
    along with the time of the maxima.
This function also applies the topographical height mask described in subsection~\ref{subsec:furthercons},
    returning an XArray DataSet of the radar data that will be used to describe and generate statistics for the event.

\subsubsection{Computing Events and Definition}

Now that the seasonal maxima have been found,
    the data can be used to define events.
This process is done in the \texttt{gen\_radar\_data} function in the \texttt{stonehavenRainLib} Python script.
The seasonal maxima,
    of which there is one for each grid cell in each year in the dataset,
    are grouped by the time in which they occurred.

These groups are 12-hour periods,
    representing either the AM or PM hours of a day.
Each group is then an event,
    containing the grid cells which experienced their maximum hourly rainfall within the group's time period.
The groups with less than 10 grid cells were discarded,
    as these events are less than 250km$^2$ in size and
    so do not represent events similar enough to the Stonehaven event.

This gives a final event definition as 250km$^2$ of the land below 400m and within +/- 50km of Stonehaven
    experiencing their annual summer hourly rainfall maxima in the same 12-hour period.
For each event, the 0.05, 0.1, 0.2, 0.5, 0.8, 0.9 and 0.95 quantiles were found and returned in a dataset.
The intensity of a given event is defined by its 0.95 quantile.

Due to the separation into 12-hour bins,
    it is assumed that each event is independent in all future calculations.

\begin{figure}[H]
    \centering
    \includegraphics[width=100mm]{dataprocessschematic}
    \caption[Diagram representing the data processing.]{
        Diagram representing the data processing after the maxima of each grid square has been found,
    within each year.
    `rain' represents the summer maximum rainfall in a grid cell,
        `time' represents the time of this maximum,
        red horizontal lines represent the taking of quantiles from the rainfall maxima.
    Number of data points and quantiles are not to scale.}
    \label{fig:dataprocessschematic}
\end{figure}

\subsubsection{Fitting event distribution}

The radar data is fit to a GEV distribution with the \texttt{comp\_radar\_fits} Python script,
    which calls the \texttt{gev\_r} Python library to run code in R .
The `fevd', fit extreme value distribution,
    command in R is used to fit an extreme value distribution to the radar data,
    which uses MLE, as described in subsection~\ref{subsec:parameterest}.
This command is applied each of the quantiles of the event,
    with the events as the data points,
    giving the parameters of GEV distributions describing each quantile of the events.
With these parameters generated,
     it was possible to find the empirical return period for the Stonehaven event.

To generate uncertainties,
    a Monte Carlo bootstrap method is used,
    given in the \texttt{mc\_dist} function.
This involved generating a list of the same size of the number of events,
    taking a random sample with replacement,
    then applying the `fevd' command to each sample.

1000 samples were taken,
    of which 9 provided anomalous (negative) location or scale parameters,
    and so were discarded.
The discarding process was valid as the value of these parameters was ~-100,
    while all other samples had values of these parameters between 2 and 15,
    as well as that removing this few samples would not have a significant effect on the 95\% confidence interval.
The 991 samples left are greater than the 800 suggested by Booth and Sarkar~\cite{Booth_Sarkar_1998}
    for an effective error estimation.

\section{Modelled Data}\label{sec:model}

The model data used in this section is
    the Met Office Hadley Centre's UKCP18 Convection-Permitting Model Projections for the UK at 2.2km resolution~\cite{model_data}.
This dataset was chosen as the resolution is less than 5km,
    and so is able to accurately represent convective storms.

The CPM ensemble consists of 12 members and covers the 100 years from 1981 to 2080,
    modelling the RCP8.5 emissions scenario.
In this investigation,
    the model's hourly temperature and rainfall data will be used

\subsection{Pre-processing}\label{subsec:preprocess}

The CPM data used is on longitude-latitude coordinates,
    with a north pole rotated to near the region the model covers.
The location of the Stonehaven crash in this alternative coordinate system was found using the \texttt{comp\_rotated\_coords} Python script,
    using CartoPy for the computations.
This coordinate rotation was also done for the four weather stations used to compute Central England Temperature (CET).

The pre-processing of the CPM data is done using the \texttt{ens\_seas\_max\_coarsen} python script,
    which finds the seasonal one-hour rainfall maxima for each grid cell.

\subsubsection{Coarsening}

The CPM data uses a 2.2km resolution,
    while the radar data uses a 5km resolution.
To account for this,
    the radar data is coarsened using the XArray \texttt{.coarsen} and \texttt{.mean} methods,
    taking the mean average of four 2.2kmx2.2km grid cells, giving 4.4kmx4.4km grid cells.

The height mask for the CPM was in the same 2.2km resolution
    and so required the same pre-processing

\subsubsection{Height mask}

To match the event definition in section~\ref{subsec:radarprocess},
    a similar height mask must be used,
    with the same limits of 0m to 400m.

The CPM used for this study is a projection over land,
    with other UKCP18 models covering the marine domain,
    albeit at a lower resolution of 12km and covering storm surges and sea levels as opposed to precipitation,
    as described in the guidance of~\cite{model_data}.
This makes the >0m height mask on the model a necessity,
    as the model projections over the sea (with height 0m) are invalid.

\subsection{Covariates}\label{subsec:covfit}

\subsubsection{Time Series}

The \texttt{comp\_cpm\_ts} Python script was used to compute the time series.
This script takes each month in each ensemble member the CPM data and computes 3 values.
These are computed using the 2.2km data without the coarsening or height mask.
First is the Central England Temperature,
    which takes the CPM data's average monthly temperature at the weather stations used to define CET,
    while the other two time series use the average monthly temperature and precipitation in the Stonehaven Region.

Two additional monthly time series were computed in the \texttt{analyse\_cpm\_data\_rolling} Python script.
One of these takes the median extreme precipitation across the Stonehaven Region
    using the extrema processed in \texttt{ens\_seas\_max\_coarsen.sh},
While another uses the temperature in the Stonehaven Region to find the Saturation Specific Humidity.

The formula used for this is the August-Roche-Magnus approximation to the Clausius-Clapeyron relation~\cite{Alduchov_Eskridge_1996}, given by:
\begin{equation}\label{eq:qsat}
    e_s = 6.112 \exp\left( \frac{17.6 T}{T + 243.5} \right)
\end{equation}
Where $T$ is the temperature in degrees Celsius.

The scaling of the Saturation Specific Humidity with temperature will be described as the Clausius-Clayperon scaling.
This is as it represents the amount of moisture that the atmosphere can hold,
    suggesting that rainfall should also scale by this amount with a change in temperature.

Each time series had 14400 members,
    with 12 ensemble members providing data for the 1200 months across the 100-year window of the model.
This was then reduced to 3600 data points by only taking the three summer months of each year.

\subsubsection{Fitting Covariates}

\texttt{analyse\_cpm\_data\_rolling} performs a linear fit of the Saturation Specific Humidity,
    Stonehaven Region temperature, mean average Stonehaven Region precipitation and mean maximal Stonehaven Region precipitation.
    using the \texttt{statsmodels} Python package's ordinary least squares regression feature.
This provides the change in each of these variables, as well as the uncertainty with $R^2$ coefficients,
    with respect to each fit.

In a method analogous to the \texttt{comp\_radar\_fits} Python script calling the \texttt{gev\_r} library to use R to fit
    a distribution to the radar rainfall extrema in subsection~\ref{subsec:radarprocess},
    \texttt{analyse\_cpm\_data\_rolling} calls the \texttt{gev\_r} library to perform fits on the model rainfall extrema.
This is done with the same `fevd' command in the extRemes R library~\cite{extremes_R},
    except done with the monthly mean CET, total model area temperature, Stonehaven Region temperature and Stonehaven Region humidity as covariates.

For this,
    the location and scale parameters are allowed to change linearly with the variable $x$.
The parameters are then $\mu = \mu_0 + x\mu_1$ and $\sigma = \sigma_0 + x\sigma_1$.
The analysis is then repeated with the shape parameter also allowed to vary in the form $\xi = \xi_0 + x\xi_1$.

In contrast to \texttt{comp\_radar\_fits},
    \texttt{analyse\_cpm\_data\_rolling} does not perform a single GEV fit on a set of events,
    but instead a GEV fit for each grid cell in the Stonehaven region for the CPM .
This far greater number of fits is possible as the CPM data contains data for 1200 summer maxima for each point,
    with each of the 12 ensembles spanning 100 years,
    as opposed to the 16 for the radar data.
A data fit for each grid cell then allows the effect of the location on the CPM rainfall maxima to be analysed.
This data fit is done nine times,
    with either no covariate or the CET, CPM Region temperature, Stonehaven Region temperature or Stonehaven Region saturated humidity as a covariate.
For each covariate, this is performed with and without the shape parameter varying.

The scaling of the parameters $\mu_1 / \mu_0$, $\sigma_1 / \sigma_0$ and, where the shape is also fit to the covariate,
    $\xi_1 / \xi_0$ can then be found at each grid cell.
These fits and scalings are also computed for the 2 hour, 4 hour and 8 hour summer rainfall maxima.

For the CET fit,
    confidence intervals in these scalings are found using a Monte Carlo bootstrap method.
This takes a random sample, with replacement, of 1\% of the grid cells and takes the mean of the scaling that sample,
    with the small sample size allowing the assumption of the samples being independent.
The quantiles of the 1000 samples taken are then used as confidence intervals.
Both the bootstrap sample means and overall mean of the scalings are saved to be used in the risk ratio calculation.
The overall mean scalings will be considered as simply the model scalings in later steps in the development.

These scalings can be compared to the scaling expected from the Clausius-Clapeyron relationship,
    which is computed by taking the linear fit of $H = H_0 + xH_1$ and computing $H_1 / H_0$.
This is a valid comparison, as if the rainfall maxima is given by the CDF~\ref{eq:gevcdf},
    then the distribution of the rainfall maxima multiplied by a constant is given by the same CDF,
    with the location $\mu$ and scale $\sigma$ multiplied by the same constant.

For both the linear and GEV fits against CET, the $\alpha_0$ for all parameters $\alpha$ is given by the value of that parameter for
    the mean summer CET from 2012 to 2021.
The covariate $x$ may then be determined as the difference from the CET in a month to the mean summer CET in 2012--2021.

\subsection{Calculating Risk Ratios}\label{subsec:riskratios}

The final calculation steps are performed in the \texttt{plot\_intensity\_risk\_ratios} Python script.
This begins by finding the value of CET in different time periods.
For historical time periods,
    this is recorded by the Met Office~\cite{CET},
    and the summer mean CET is used.
The periods this data is drawn from are 1850--1899 (Pre-Industrial),
    1980--1989, 2005--2020 and 2012--2021.
For the hypothetical 2K warmer than Pre-Industrial word seasonal mean CET,
    the Pre-Industrial CET is increased by 1.88$\pm$0.06 degrees using a normal distribution,
    taking the distribution from~\cite{Tett_Soon}.

\subsubsection{Applying CPM scaling to Empirical Distribution}

Let $T_0$ be the mean summer CET in the 2005--2020 range.
For each temperature, $\delta T$ is computed as $T - T_0$.

The value of a parameter $\alpha$ in the GEV distribution fit to empirical events is then scaled to $\alpha_T$ at the new temperature:
\begin{equation}\label{eq:newradarparams}
    \alpha_T = \alpha + \delta T \frac{\alpha_1}{\alpha_0} \alpha
\end{equation}
Where $\alpha_1$ is the change in the parameter of the distribution of the extreme events in the model with respect to CET and
    $\alpha_0$ is the value of the parameter of the modelled extreme event distribution in the 2012--2021 temperature range.

This process is repeated, once with the shape parameter fixed, only applying the scalings $\mu_1 / \mu_0$ and $\sigma_1 / \sigma_0$
    from the model distribution to the empirical distribution and again also applying the third $\xi_1 / \xi_0$ from the shape covariate fit.
Both of these are repeated four times, using the scalings from the 1, 2, 4 and 8 hour model maxima,
    although only the 1 hour maxima will be used to directly compute any risk ratios.

This changes the empirical event distribution equivalently to the fitting of covariates to the model distribution in subsection~\ref{subsec:covfit},
    as where the model parameters are multiplied by a factor $k$ due to changes in CET,
    the event distribution parameters are assumed to also be multiplied by a factor of $k$.
The key assumption made here is that the processes driving the change in the maxima from the model will act equivalently
    on the extreme events defined in subsection~\ref{subsec:radarprocess}, as discussed in a supplement to~\cite{Tett_Soon}.

\subsubsection{Calculating Ratios}

To calculate the probability ratio for an event of a given intensity $I$,
    the survival function of the distribution at each time period is used to find the probability of finding an event of equal or greater intensity.
$p_0$ is this probability for the distribution in the Pre-Industrial (1850--1899) world,
    with $p_1$ being the survival function value for a given other time period.
The probability ratio for the event of intensity $I$ in the time period is then $p_1/p_0$.

Equivalently, to calculate the intensity ratio for an event with return period $P$,
    the inverse survival function of the distribution at each time period is used to find the event for which the probability of finding an event of equal to or greater intensity is $1/P$.
$I_0$ and $I_1$ are the intensities of the event for the distribution in the Pre-Industrial world and in the given time period respectively,
    with the intensity ratio as $I_1/I_0$.
An intensity ratio expected from Clausius-Clapeyron scaling was also computed directly by allowing the event intensity to scale identically to saturation humidity with temperature.

As both the intensity and the return period of the Stonehaven event are known,
    it is straightforward to calculate these ratios.
The `headline' ratios found in the abstract are those found by applying the 1 hour model maxima scaling to the empirical event distribution.
However, the ratios were also determined for a collection of intensities and return periods to allow the effect of these on the ratio to be plotted.
This is repeated for using fits based on the 1, 2, 4 and 8 hour model maxima.

For the probability and intensity ratios found for the fits based on the 1 hour model maxima,
    comparing Pre-Industrial to a 2K warmer world,
    a Monte Carlo bootstrap is used to find confidence intervals.
This takes \textasciitilde1000 random samples, each taking one element of the set of \textasciitilde1000 empirical fits found from the Monte Carlo process in subsection~\ref{subsec:radarprocess}
    and one element of the set of \textasciitilde1000 CPM scalings found in the Monte Carlo process in subsection~\ref{subsec:covfit}.
Each sample applies its CPM scaling to its Distribution and calculates the risk ratios as described above.
The quantiles of these probability and intensity ratios give the confidence intervals.

The computation of the probability and intensity ratios is repeated with alternative definitions of the intensity of the event
    as each of 0.05, 0.1, 0.2, 0.5, 0.8 and 0.9 quantiles as opposed to the 0.95 quantile from the definition in subsection~\ref{subsec:radarprocess},
    with the emprical fit for each of these quantiles scaled to the 1 hour model maxima,
    to assess the effect of event definition on the outcome of the analysis.
The risk ratio analysis is done twice, once with and once without a covariate fit shape parameter.

\chapter{Results and Analysis}\label{ch:results}
\section{Results}\label{sec:results}

\begin{comment}
This section should detail the obtained results in a clear,
easy-to-follow manner. It is important to make clear what are original
results and what are repeats of previous calculations or computations.
Remember that long tables of numbers are just as boring to read as
they are to type-in!

Use graphs to present your results wherever practicable.

Results or computations should be presented with uncertainties
(errors), both statistical and systematic where applicable.

Be selective in what you include: half a dozen \emph{e.g.}~tables that
contain wrong data you collected while you forgot to switch on the
computer are not relevant and may mask the correct results.
\end{comment}

\subsection{Radar Data Fit}\label{subsec:radardatafit}

\begin{figure}[H]
    \centering
    \includegraphics[width=150mm]{radarreturnplot}
    \caption{A line graph plotting the return period in years against the intensity of the event,
        taken from fitting a GEV distribution to the event distribution.
    The dashed purple vertical line is the intensity of the Stonehaven event.
    The solid red vertical line is the actual one-hour rainfall maximum at the Stonehaven crash location.
    The grey area gives the uncertainties from a Monte Carlo bootstrap of approximately 1000 samples.}
    \label{fig:radarreturnplot}
\end{figure}

It was found that the maximum hourly rainfall at the crash location was 18.2mm/hr and
    that the empirical return period for the Stonehaven event was 17 years,
    as can be seen in figure~\ref{fig:radarreturnplot}.
The intensity of the Stonehaven event,
    the 0.95 quantile of the radar grid cells having their 2020 seasonal maxima in the AM of the 12th of August,
    was found to be 25.2mm/hr.

Figure~\ref{fig:radarreturnplot} was created using the survival function of the GEV distribution fitted to the event data,
    getting the return period as the reciprocal of the probability.
This had parameters location $\mu = 10.7$, scale $\sigma = 4.26$ and shape $\xi = -0.173$.

\subsection{Model Data}\label{subsec:modelcorr}

\subsubsection{Model Correlations}

\begin{figure}[H]
    \centering
    \includegraphics[width=150mm]{2cet_scatter}
    \caption{Scatter diagrams, from the Convection Permitting Model,
        for summer CET vs Stonehaven region saturated humidity (g/Kg) (a),
        Regional Temperature (b),
        Regional  Precipitation (mm/day) (c) and
        spatial median Rx1hr (mm/hr) for the region.
    The region is all points within a square of about 100x100km centered on the Stonehaven crash location.
    Colors indicate the ensemble number, black line is the best-fit regression slope.
    Text box shows best estimate and standard error in the estimated regression.
    Dotted black vertical lines show (left to right) CET summer mean for 1850--1899, 2021 and estimated CET at +2K warming.
    Title shows $R^2$ correlation coefficients for fit.
    Dotted vertical red lines show minimum and maximum CET values from 1850--2020 observations.}
    \label{fig:2cet_scatter}
\end{figure}

It is unsurprising that the correlations found in (a) and (b) are very similar,
    with both having $R^2$ values that round to the same number to the nearest percentage.
Some data points appear in almost identical points in both scatter diagrams.
This is evident from equation~\ref{eq:qsat}.
The by differentiating at $T=18$,
    equation~\ref{eq:qsat} has Taylor series to first order of $20.5627+1.2863(T-18)$.
By substituting $T=14$ and $T=22$ into this linear approximation,
    only small errors of $0.532$ and $0.603$ are found.
This suggests that in the CET range, equation~\ref{eq:qsat} is effectively linear and so the fit onto the Stonehaven Region saturated humidity
    is a linear scaling of the fit onto the Stonehaven Region Temperature.

\subsubsection{Model GEV fit}

\begin{figure}[H]
    \centering
    \includegraphics[width=100mm]{2cpm_gev_fit}
    \caption{a) Location parameter $\mu = \mu_0 + \overline{T_{2012-2021}}\mu_1$  for 2012--2021 CET and Rx1hr.
    b) Scale parameter $\sigma = \sigma_0 + \overline{T_{2012-2021}}\sigma_1$ for 2012--2021 CET and Rx1hr.
    c) Scatter plot of d(location)/d(CET) vs d(scale)/d(CET) as of 2012--2021 parameters.
    Dashed lines show expected changes if changes predicted by Clausius-Clapeyron relationship.
    Dots are colored by topography.
    Large red dot shows mean over all points where height is more than 0m and less than 400m.
    Red ellipse shows 95\% uncertainty for this mean, see subsection~\ref{subsec:riskratios}.
    Other colored large dots and ellipses show similar means and uncertainty ellipses for Rx2hr (brown), Rx4hr (purple) and Rx8hr (blue).
    Black ellipse shows 95\% uncertainty for all Rx1hr points where height is less than 0m and more than 200m.
    d) as c) but for case with shape parameter varying with CET
    e) QQ-plot for GEV fit to CPM data using CET as a covariate at the Stonehaven Crash location.
    f) as e) but for point 25km west of the Stonehaven Crash Location.}
    \label{fig:2cpm_gev_fit}
\end{figure}
\begin{table}[H]
    \centering
    \begin{tabular}{c c c c c}
        Covariate\textbackslash Data & Rx1h & Rx2h & Rx3h & Rx4h \\
        None &7498&6533&5341&4130 \\
        CET &7433&6482&5305&4109 \\
        CET w/ Shape &7434&6483&5305&4109 \\
        CPM Region &7402&6457&5287&4097 \\
        CPM Region w/ Shape &7403&6458&5287&4097 \\
        Stonehaven Region &7402&6457&5287&4098 \\
        Stonehaven Region w/ Shape &7403&6458&5288&4098 \\
        Stonehaven Humidity &7405&6459&5289&4099 \\
        Stonehaven Humidity w/ Shape &7406&6460&5290&4100 \\
    \end{tabular}
    \caption{AIC for maximum summer 1, 2, 4 and 8 hourly summer maximum rainfall
        above 0m and below 400m in the Stonehaven Region (+/- 0.5 degrees of the Stonehaven Crash location)
        GEV fits with different covariates.
    The covariates are CET, the average temperature in the CPM region (UK),
    the average temperature in the Stonehaven Region and
    the Saturation Humidity in the Stonehaven Region}
    \label{tab:AICtable}
\end{table}
\begin{table}[H]
    \centering
    \begin{tabular}{c c c}
        Dataset\textbackslash Parameter  & Location $\mu$ & Scale $\sigma$ \\
        Rx1hr &0.98&1.92 \\
        Rx2hr &0.80&1.67 \\
        Rx4hr &0.59&1.33 \\
        Rx8hr &0.38&1.01 \\
    \end{tabular}
    \caption{Ratio of mean parameter scalings to Clausius-Clapeyron scaling, $\left( \frac{\alpha_1}{\alpha_0} \right) / \left( \frac{H_1}{H_0} \right)$,
        for parameter $\alpha$ and saturated humidity $H$,
        where the shape parameter is fixed.
        Subscript $0$ is the value in the average summer CET of 2012--2021,
            subscript $1$ is the increase for every one degree increase of CET.
    Clausius-Clapeyron scaling is 5.73\%.}
    \label{tab:CCtable}
\end{table}


\subsection{Risk Ratios}\label{subsec:riskratio}

\begin{figure}[H]
    \centering
    \includegraphics[width=\linewidth]{2probradarcpm}
    \caption{Intensity ratio increase (\%) as a function of the return period (a) and
    probability ratio as a function of regional Rx1hr (b).
    The best estimate (line) and 5--95\% uncertainty range (shading) are shown.
    Horizontal dotted lines in a) show expected intensity change if extremes scale with Clausius-Clapeyron.
    The top axis in a and b shows the equivalent rainfall and return period estimated from the radar rainfall data while
    vertical purple line shows the regional rainfall maximum for 2020.
    Dotted, dashed and dot-dashed lines in a and b show results when scaling estimated from simulated 2, 4 and 8 hourly summer maxima (Rx2hr, Rx4hr and Rx8hr).
    c) Intensity ratio increase (\%) as a function of return period for best estimates using different quantiles (labels on PI+2K line) to define regional extreme.
    Hexagon markers show 95\% quantile used in a and b.
    d) As c) but for probability ratio as a function of Rx1hr.
    Vertical purple line shows Stonehaven Crash Rx1hr for 2020-08-12.
    Note these are different from the regional 95\% quantiles shown in a and b.}
    \label{fig:2probradarcpm}
\end{figure}

\begin{figure}[H]
    \centering
    \includegraphics[width=\linewidth]{2probradarcpmshape}
    \caption{Identical to figure~\ref{fig:2probradarcpm},
    with the shape parameter also allowed to vary with CET.}
    \label{fig:2probradarcpmshape}
\end{figure}

\begin{table}[H]
   \centering
    \begin{tabular}{c c c}
        Time Period & Intensity Ratio (\%) & 5--95\% uncertainty (\%) \\
        1980s & 2 & 2--3 \\
        2012--2021 & 9 & 11--7 \\
        PI+2K & 18 & 14--22 \\
        With varying shape: && \\
        1980s & 2 & 2--3 \\
        2012--2021 & 9 & 11--7 \\
        PI+2K & 18 & 13--22 \\
    \end{tabular}
    \caption{A table with the intensity ratios $I_1/I_0$ for three different time periods,
        both with and without a covariate shape parameter.
    $I_1$ is the intensity of the event in the Time Period.
    $I_0$ is the intensity of the event Pre-Industrial (1850-1899).}
    \label{tab:irtable}
\end{table}

\begin{table}[H]
   \centering
    \begin{tabular}{c c c}
        Time Period & Probability Ratio (\%) & 5--95\% confidence interval (\%) \\
        1980s & 1.10 & 1.06--1.15 \\
        2012--2021 & 1.37 & 1.22--1.61 \\
        PI+2K & 1.78 & 1.46--2.34 \\
        With varying shape: && \\
        1980s & 1.10 & 1.06--1.15 \\
        2012--2021 & 1.37 & 1.23--1.62 \\
        PI+2K & 1.79 & 1.47--2.38 \\
    \end{tabular}
    \caption{A table with the probability ratios $p_1/p_0$ with confidence intervals for three different time periods,
        both with and without a covariate shape parameter.
    $p_1$ is the probability of the event in the Time Period.
    $p_0$ is the probability of the event Pre-Industrial (1850-1899).}
    \label{tab:prtable}
\end{table}

\begin{table}[H]
    \centering
    \begin{tabular}{c c c}
        Time Period\textbackslash Parameter & Location $\mu$ & Scale $\sigma$ \\
        PI & 10.2 & 3.90 \\
        1980s & 10.4 & 4.02 \\
        2012--2021 & 10.8 & 4.34 \\
        PI+2K & 11.3 & 4.79
    \end{tabular}
    \caption{A table showing the parameters of the GEV distribution for the intensity an event defined in subsection~\ref{subsec:radarprocess} after
        applying the scaling of the location $\mu$ and scale $\sigma$ parameters to the temperatures of different time periods.}
    \label{tab:0.95params}
\end{table}

The data in table~\ref{tab:0.95params} does not have a covariate shape,
    using the fixed shape parameter $\xi = -0.173$ from the empirical fit.
The shape parameter is in the $-0.174$--$-0.172$ range for all three fits with covariate shape,
    so it effectively duplicates the parameters here.

As the shape parameter is negative,
    it is possible to compute hypothetical the maximum,
    infinite return period intensity of an event using the formula $\mu - \frac{\sigma}{\xi}$.
This gives 32.7, 33.6, 35.9 and 39.0mm/hr as hypothetical maxima in
    PI, the 1980s, 2012--2021 and PI+2K Time Periods respectively.

\begin{figure}[H]
    \centering
    \includegraphics[width=150mm]{scaledgevdists}
    \caption{A line chart of the Probability Density Functions of the GEV distributions with parameters given in table~\ref{tab:0.95params},
        with a shape parameter $\xi = -0.172$.
    The Time periods are Pre-Industrial (PI), the 1980s, 2012-2021
    X-axis gives the 0.95 quantile of the maximum summer rainfall by grid cell,
        Y-axis gives the probability density in probability/Rx1hr.
    Vertical purple line is the intensity of the Stonehaven Event.}
    \label{fig:scaledgevdists}
\end{figure}

Figure~\ref{fig:scaledgevdists} illustrates the probability ratios in the top half of table~\ref{tab:prtable}.
$p_0$ is the area to the right of the purple line between 0 and the blue curve,
    while $p_1$ is the area to the right of the purple line between 0 and the yellow, green and red curves for
    the 1980s, 2012--2021 and a world 2K warmer than pre-industrial respectively.


\section{Discussion}\label{sec:discussion}

\begin{comment}
This section should give a picture of what you have taken out of your
project and how you can put it into context.

This section should summarise the results obtained, detail conclusions
reached, suggest future work, and changes that you would make if you
repeated the project.
\end{comment}

\subsection{Event Definition}\label{subsec:diseventdef}

%  TODO mention lack of trend analysis due to lack of radar data

%  TODO mention Koppen in analysis

\subsection{Model Resolution}\label{subsec:dismodeldef}

Mention~\cite{Kendon_Fischer_Short_2023}
%  TODO insert chart w various coarsenings

\subsection{Model Validity}\label{subsec:dismodelvalid}

\chapter{Conclusions}\label{ch:conclusions}

\begin{comment}
This is the place to put your conclusions about your work. You can
split it into different sections if appropriate. You may want to include
a section of future work which could be carried out to continue your
research.

The conclusion section should be at least one page long, preferably 2
pages, but not much longer.
\end{comment}

\appendix
% the appendix command just changes heading styles for appendices.

%  TODO insert code behind each figure

\chapter{Distribution of events per year}

\begin{figure}[H]
    \centering
    \includegraphics[width=150mm]{eventsyear}
    \caption[A bar chart showing the number of events occurring in each year.]{A bar chart showing the number of events,
        that being 10 grid cells taking their maximum summer rainfall value within the same 12-hour bin,
    occuring within each year.}
    \label{fig:eventsyear}
\end{figure}
\begin{figure}[H]
    \centering
    \includegraphics[width=150mm]{cellsyear}
    \caption[A bar chart showing the number of cells that contribute to events.]{
        A bar chart showing the number of cells that contribute to events,
        that being having their summer rainfall maxima in the same 12-hour bin as at least 9 other cells.
        The theoretical maximum is 185 cells.}
    \label{fig:cellsyear}
\end{figure}


\bibliographystyle{unsrt}
\bibliography{ref}




\end{document}

