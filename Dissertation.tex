\documentclass[12pt,a4paper]{report}

\usepackage{graphics}
\usepackage{fullpage,epsf,graphicx}%, amstext,url} 
\graphicspath{{figures/}}
\def\BibTeX{{\rm B\kern-.05em{\sc i\kern-.025em b}\kern-.08em
    T\kern-.1667em\lower.7ex\hbox{E}\kern-.125emX}}

\begin{document}

\thispagestyle{empty}

%
%	This is a basic LaTeX Template for the TP/MP MSc Dissertation report

\parindent=0pt          %  Switch off indent of paragraphs 
\parskip=5pt            %  Put 5pt between each paragraph  

%	This section generates a title page
%       Edit only the sections indicated to put in the project title, and submission date

\vspace*{0.1\textheight}

\begin{center}
        \huge{\bfseries Event Attribution}\\ % Replace with the title of your dissertation!
\end{center}

\medskip

\begin{center}
        \Large{Christopher Long}\\  % Author of dissertation - replace with your name!
        \medskip
        \large{August 18, 2023}  % Submission date
\end{center}

%%% If necessary, reduce the number 0.4 below so the University Crest
%%% and the words below it fit on the page.
%%% Don't let the crest, or the wording below it, flow onto the next page!

\vspace*{0.4\textheight}

\begin{center}
        \includegraphics[width=35mm]{crest.pdf}
\end{center}

\medskip

\begin{center}

\large{
  MSc in Mathematical Physics\\[0.8ex]
  The University of Edinburgh\\[0.8ex]
  2023}

\end{center}

\newpage


\pagenumbering{roman}

\begin{abstract}
  Heavy rainfall on the 20th of August 2020 resulted in a landslip on the Dundee-Aberdeen railway line near Carmont, Aberdeenshire.

  This caused the Stonehaven derailment,
    in which three people were killed.

  In this dissertation,
    I quantify the effect of observed and future climate warming on the risk of similar extreme rainfall events occuring.



\end{abstract}

\pagenumbering{roman}

\begin{center}
\textbf{Declaration}
\end{center}

I declare that this dissertation was composed entirely by myself.

Give the chapters and state who did the work

\bigskip

Calculations were done using Python with use of various packages.

Numpy and XArray are used to manage and process both real and modelled rainfall data.

Cartopy is used to process data between both Ordinance Survey and Longitude-Latitude coordinate systems.

R and the RPy package are used to fit extreme value distributions to the rainfall data.

The data processing ran on both the JASMIN supercomputer and on my local machine,
GitHub was used to manage the transfer of both code and data.

The majority of the code used in this project is based on the code used by my supervisor performing a similar analysis on the 2021 Edinburgh cloudburst.


\newpage

\begin{center}
\textbf{Personal Statement} % Use Git commits
\end{center}

I spent the first 2 weeks of the project retro-fitting the Edinburgh castle
library to apply to the Stonehaven crash.
This process began by creating plots including the
rainfall at the crash location on the day and an animation of the rainfall
in the Stonehaven region, allowing me to find an appropriate event definition.

For the remaining weeks of June, I implemented a topographical height mask
to the radar data, as well as establish a workspace to transfer data between
my laptop and the JASMIN supercomputer.

Following this I began my analysis of the radar data.
This involved computing and plotting the monthly rainfall maxima in
the Stonehaven region, as well as finding the quantiles of these extrema,
as well as using R to find the parameters of a GEV fit to both the empirical
and a monte carlo bootstrap of the rainfall distribution.

In mid-July I created plots for these distributions and started to work with the
Convective Permitting Model data.

For the final parts of the summer I worked mainly on perfecting the
algorithm for the Random Osculator and implementing the EvenMoreRandom
Osculators algorithm with the improved Heaviside Articulation. The
final results were encouraging, but more work is clearly needed. To
this end, I have been awarded a studentship by the British University
of Lifelong Learning to extend this work during my PhD Studies at
the non-existent Scottish Highlands Institute of Technology in
Inveroxter.

I started writing this dissertation in mid-July, and I spent the first
three weeks of August working on it full-time.



\newpage

\begin{center}
%\vspace*{2in}
% an acknowledgements section is completely optional but if you decide
% not to include it you should still include an empty {titlepage}
% environment as this initialises things like section and page numbering.
\textbf{Acknowledgements}
\end{center}

I'd like to thank my supervisor Professor Simon Tett for
making this project possible



\tableofcontents
\listoftables
\listoffigures

\pagenumbering{arabic}

\chapter{Introduction}
The Introduction should contain a description of your project and the
problem you are trying to solve. It should start off at a level that
should be understandable by anyone with a degree in physics, but it
can become more technical later

Where appropriate you should include references to work that has
already been done on your topic and anything else which lets you set
your work in context.

One of the things you will need to do is to ensure that you have a
suitable list of references.  To do this you should see \cite{ref:lam}
or some other suitable reference.  Note the format of the citation used
here is the style favoured in this School.  Here is another
reference \cite{ref:bloggs} for good measure.

Alternatively, you can use \BibTeX. See later for some details on this.

You will also want to make sure you have no spelling or grammatical
mistakes. To help identify speling mistukes you can use the commands
\emph{aspell}, \emph{ispell} or \emph{spell} on most Linux/Unix
computers. See the appropriate manual pages. Remember that spelling
mistakes are not the only errors which can occur. Spelling checkers
will not find errors which are, in fact, valid words such as
\emph{there} for {\em their}, nor will they find repeated repeated
words which sometimes occur if your concentration is broken when
typing. \textbf{There is no substitute for thorough proof reading!}

Your dissertation should be no longer than 15,000 words. In terms of
pages, 30 pages are ok. 50 pages are fine. But it shouldn't be
much longer than that.


\chapter{Event Attribution Theory}
\chapter{Extreme Value Statistics}


\chapter{Design and/or development (of my project)}

This section should be written in standard scientific
language. Standard techniques in your research field should not be
written out in detail. In computational projects this section should
be used to explain the algorithms used and the layout of the
computational code. A copy of the actual code may be given in the
appendices if appropriate.

This section should emphasise the philosophy of the approach used and
detail novel techniques. However please note: this section should not
be a blow-by-blow account of what you did throughout the project. It
should not contain large detailed sections about things you tried and
found to be completely wrong! However, if you find that a technique
that was expected to work failed, that is a valid result and should be
included.

Here logical structure is particularly important, and you may find
that to maintain good structure you may have to present the
explorations/calculations/computations/whatever in a different order
from the one in which you carried them out.



\chapter{Another Chapter Title}
\section{Number of Chapters}

You may vary the number of chapters. The Introduction and Background
Theory chapter are essential, although you may choose a different
title for the latter. These two introductory chapters are usually
followed by a chapter on what you did yourself, with a title such as
Design and Development, although you can choose any title you
wish. After that, you might to have another chapter, or you may go
straight to the Results and Conclusions chapter.

After the Introduction, you are free to use any chapter titles you wish.




\chapter{Results and Analysis}

This section should detail the obtained results in a clear,
easy-to-follow manner. It is important to make clear what are original
results and what are repeats of previous calculations or computations.
Remember that long tables of numbers are just as boring to read as
they are to type-in!

Use graphs to present your results wherever practicable.

Results or computations should be presented with uncertainties
(errors), both statistical and systematic where applicable.

Be selective in what you include: half a dozen \emph{e.g.}~tables that
contain wrong data you collected while you forgot to switch on the
computer are not relevant and may mask the correct results.

\section{Discussion of your results}

This section should give a picture of what you have taken out of your
project and how you can put it into context.

This section should summarise the results obtained, detail conclusions
reached, suggest future work, and changes that you would make if you
repeated the project.

\chapter{Conclusions}

This is the place to put your conclusions about your work. You can
split it into different sections if appropriate. You may want to include
a section of future work which could be carried out to continue your
research.

The conclusion section should be at least one page long, preferably 2
pages, but not much longer.

\appendix
% the appendix command just changes heading styles for appendices.

\chapter{Stuff that's too detailed}

Appendices should contain all the material which is considered too
detailed to be included in the main body of the text, but which is
important enough to be included in the thesis.

Perhaps this is a good place to mention \BibTeX.

You can do references in the simple way explained in the introduction,
or you can use \BibTeX.


\section{\BibTeX}
\label{sec:bibtex}

It is convenient to use \BibTeX\ to compile your bibliography.  First
you need to create a .bib file e.g.  you may call it ref.bib Then you
can put all your references into the file with entries such as
\begin{verbatim}
@Book{ob:bornwolf,
     author = "Born, M and Wolf, E",
     title  = "Principles of Optics",
     publisher = "Cambridge University Press",
     year = 1999,
     edition = {7th},
}

@Article{jr:ashkin,
Author = {A. Ashkin and J.M. Dziedzic and J.E. Bjorkholm and S. Chu},
Title = "Observation of a single beam gradient force optical tap for 
dielectric particles",
Journal = "Optics Letters",
Volume = 11,
Pages = "288-290",
Year = 1986}

@INPROCEEDINGS{seger,
 author = {J. Seger and H.J. Brockman},
 title = {What is bet-hedging?},
 editors={P.H. Harvey and L. Partridge},
 booktitle = {Oxford Surveys in Evolutionary Biology},
 year={1987},
 page={18},
 publisher={Oxford University Press},
 place={Oxford}}
\end{verbatim}
for a book, an article in a journal or an article in a proceedings volume
respectively.

Inside your \LaTeX\ file
you should include 
\begin{verbatim}
\bibliographystyle{unsrt}                      
and
\bibliography{ref}
\end{verbatim}
The first command determines the reference style, here plain and 
unsorted. With this referencing style 
a numerical referencing system (which is now the most
common in physics literature) is used and the numbering of references
will be the order in which they appear in the document. Alternatively, 
you could use
a customised `style file' but there is no real need.  The second
command just inputs your .bib file Note that only the references cited
in the text will appear in the bibliography so you can have spare
references in your .bib file.


You use the name you have given to an entry (e.g.
for the book example above the name is ob:bornwolf)
to cite the relevant article
by using the cite command in your \LaTeX\ file e.g. 
\begin{verbatim}
\cite{ob:bornwolf}
\end{verbatim}


\section{Producing your documents using \texttt{pdflatex}}

To use pdflatex your figures need to be in pdf format.  You can convert almost any image file to pdf using \texttt{convert}.  e.g. \texttt{convert myfigure.png myfigure.pdf}.

The first time you should type:
\begin{verbatim}
  pdflatex ProjectReport
  bibtex ProjectReport
  pdflatex ProjectReport
  pdflatex ProjectReport
\end{verbatim} 
This first time you run\texttt{pdflatex} it will produce a
\texttt{ProjectReport.aux}.  The \BibTeX\ command reads in the
bibliography file and makes the files \texttt{ProjectReport.bbl} and
\texttt{ProjectReport.blg} files.  These files are read in the next
\texttt{pdflatex} command, but you'll still have ``undefined
cross-reference'' errors which are sorted out by the last
\texttt{pdflatex} command.

Subsequently, you should only need to do one (or two)
\texttt{pdflatex}s, or \texttt{pdfbibtex} followed by
\texttt{pdflatex} twice if you change any references.

\vspace{5mm} You may also use plain \texttt{latex} instead of
\texttt{pdflatex}.  This requires you to use postscript graphics
instead of pdf.




\chapter{Stuff that won't be read by anyone}

Some people include in their thesis a lot of detail, particularly lots
of tables containing raw results, figures of intermediate results, or
computer code which no-one will ever read. You should be careful that
anything like this you include should contain some element of
uniqueness which justifies its inclusion.

\begin{thebibliography}{100}

    \bibitem{ref:event} G~J~van Oldenborgh, K~van der Wiel, S~Kew, S~Philip, F~Otto, R~Vautard, A~King, F~Lott, J~Arrighi, R~Singh and M~van Aalst. \emph{2021 Pathways and pitfalls in extreme event attribution.} Climactic Change, Volume 166, Article 13.

    \bibitem{ref:lam} L~Lamport. \emph{1986 \LaTeX\  User's Guide and Reference Manual.} Addison Wesley, pp~242.

    \bibitem{ref:bloggs} F~Bloggs. \emph{1993 \LaTeX\  Users do it in Environments.} International~Journal of Silly Findings, Volume 22, pp~23-29.

    \bibitem{ref:no-one} P~Thrower. \emph{2019 Too Much Rhubarb Crumble.} Dessertation Review Letters, Volume 2, pp~1-20.

    \bibitem{ref:the-early-ones} H~McDonald, S~Simpson, S~Ross and K~Green. \emph{The Ones That Got Away.} Unclear Physics, Volume 1, pp~1-68.



\end{thebibliography}


\end{document}

